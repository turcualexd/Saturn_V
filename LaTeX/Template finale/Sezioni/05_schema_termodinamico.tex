\section{Schema termodinamico}
\label{sec:schema termodinamico}

\subsection{Analisi dello schema semplificato del sistema motore}
Lo schema termodinamico semplificato del sistema propulsivo F-1 viene presentato di seguito. Per poter trattare le principali grandezze termodinamiche quali sono la pressione P, la temperatura T e la portata massica $ \dot{m} $, si è deciso di consultare i manuali del motore per poter estrapolare uno schema semplificativo a blocchi \cite{engine_manual}. Nello schema presentato viene introdotto il sistema a ciclo generatore di gas che permette l’alimentazione della turbopompa. Supponendo il funzionamento a regime (Main Stage), l’alimentazione è completamente auto sostenuta finchè non viene soppressa dai computer di bordo (al termine del $ t_{burn} $) o viene esaurito il propellente.

Qualitativamente, dai due serbatoi di LOX e RP1 viene spillata una portata, che viene trattata dalla turbopompa che porta in pressione i due liquidi. I due tank sono messi leggermente in pressione da un gas inerte: elio (e GOX nel tank LOX). Il motivo per cui si preferisce avere un gas in pressione è che permette una uscita facilitata dai due tank ed evita la cavitazione man mano che vengono svuotati i serbatoi. La turbopompa sarà trattata in dettaglio nei paragrafi successivi, data la sua complessità costruttiva. Per questo schema è sufficiente sapere che essa ha il compito di portare ad una certa pressione i due liquidi. Per poter alimentare le pompe, viene calettata sullo stesso asse in comune una turbina. Questa turbina viene mossa da dei gas caldi combusti in una piccola camera di combustione. Questo sottosistema viene chiamato Gas Generator o GG, viene alimentato da una portata spillata dopo le turbopompe della stessa coppia RP1/LOX con un eccesso di combustibile per evitare temperature elevate in ingresso turbina. I gas caldi in uscita dal GG vengono ulteriormente sfruttati per poter scaldare e quindi pressurizzare l’elio, successivamente tali gas di scarico vengono posti in uno tubo circonferenziale all’ugello nella posizione 10:1 di espansione delle aree del divergente, dove vengono scaricati sulla parete interna dell’estensione dell’ugello. Questo viene fatto per creare un film di gas relativamente freddi che hanno il compito di alleviare il carico termico sopportato da questa porzione di ugello (vedi appendice per rappresentazioni grafiche dettagliate). Il raffreddamento della parte superiore dell’ugello viene effettuato facendo passare il combustibile in diversi tubi esterni posti nella sezione tra gola e divergente 10:1, il combustibile dopo aver assorbito calore viene introdotto in camera di combustione.

In tabella e figura vediamo alcuni dati rappresentativi del sistema intero. Si è ipotizzato di trattare i gas come perfetti e di assumere come dati alcuni rendimenti e alcune grandezze caratteristiche.

\cfig{schema_termodinamico}{Schema termodinamico}{schema_termodinamico}{0.45}

\subsection{Analisi sulla scelta del ciclo di alimentazione}

Per via della presenza di un sistema a turbopompe per l'alimentazione del propellente, si deve considerare quale sia il design ottimale per il ciclo di potenza che alimenterà la turbina e di conseguenze le due pompe stesse. La scelta del tipo di ciclo di potenza ha ripercussioni sulla filosofia del design dell'intero impianto e la sua introduzione può implicare una variazione in termini di prestazioni del sistema motore. Nel motore F-1 è stato scelto un ciclo di alimentazione a Gas Generator, tra tutti i cicli possibili è sicuramente il meno complesso da trattare. E' il sistema più leggero tra tutti, e data la ridotta complessità è il più economico in termini di sviluppi. E', inoltre, l'unico ciclo di alimentazione tra i principali ad avere un flusso di gas in parallelo alla camera di spinta. Questa peculiarità, come vedremo, ha ripercussioni sulle prestazioni del motore. Altri principali tipi di cicli di alimentazione sono: l'expander cycle e lo staged combustion cycle. Oltre a questi, diversi ne sono stati sviluppati con combinazioni di diverse complessità che però in alcuni casi hanno portato a migliorare notevolmente le prestazioni globali del sistema motore.

Nel ciclo expander la turbina viene alimentata grazie al riscaldamento rigenerativo del fuel attraverso le pareti dell'ugello (permettendo quindi anche il raffreddamento delle pareti dell'ugello). Il fuel, che viene vaporizzato, espande in turbina e successivamente entra in camera di combustione. La temperatura in entrata alla turbina è limitata, non ci sono combustioni prima di essa: questo limita l'energia ricavabile dalla turbina e limita anche la pressione in camera che posso ottenere. 
Il necessario cambio di fase è un fattore limitante nell'utilizzo di questo ciclo: esso viene utilizzato per motori a spinte non troppo elevate per via del fatto che, crescendo la dimensione del motore, si debba aumentare la portata richiesta in turbina per aumentare la potenza e questa portata è limitata da fattori geometrici, poichè deve passare nelle pareti dell'ugello ed essere adeguatamente portata a vaporizzazione. Inoltre, la necessità di avere vaporizzazione immediata concentra l'utilizzo di questo ciclo su fuel criogenici come l'idrogeno. Tale ciclo quindi non poteva essere applicato al boost stage del primo stadio SI-C. Questo ciclo infatti viene utilizzato principalmente su stadi alti in cui l'espansione avviene nel vuoto, per via del fatto che il ciclo stesso limita la pressione in camera di combustione.

Nei diversi cicli di tipo staged-combustion si introduce uno o più preburner e in questo modo si suddividono diverse zone in cui avviene la combustione, oltre alla camera di spinta principale. Questo ciclo ha migliori prestazioni rispetto al ciclo a gas, poichè il flusso in uscita non viene scaricato in atmosfera ma introdotto nella camera principale. Questo aumenta l'efficienza del sistema, per contro lo sviluppo di questo ciclo aumenta notevolemente i costi e i tempi di sviluppo data la sua elevata complessità. Per poter comprendere le difficoltà ingengeristiche da affrontare si deve riconoscere che in un ciclo chiuso non si potrebbe semplicemente scaricare i gas combusti del GG, dopo aver attraversato la turbina, direttamente in camera per diversi motivi. Tra cui: la pressione troppo bassa in uscita dalla turbina non sarebbe compatibile con l'alta pressione richiesta dalla camera di spinta, prodotti di combustione di una miscela FR di combustibili a idrocarburi come RP-1 provocherebbero difficoltà agli iniettori del piatto principale, intasandoli. Da notare inoltre che nel ciclo chiuso FR tutto il fuel passa nel preburner (da qui anche il termine preburner), mentre nel gas generator venivano spillate delle piccole portate di fuel e oxidizer. Questo implica che il sistema di preburner deve essere abbastanza grande da contenere una combustione controllata e generalmente a portate elevate di propellenti (e pressioni molto alte), soprattuto per quanto riguarda i booster stage, ovvero lo stadio iniziale. Un altro accorgimento è la suddivisione in stadi della turbopompa del fluido che viene spillato (nel caso sotto dell'ossidante). L'ossidante in quel caso deve essere pressurizzato ad alte pressione per la camera e una sua piccola percentuale deve essere portata a quasi il doppio del valore di pressione della camera, per entrare nel preburner. Oltre a queste complicazioni, il ciclo chiuso FR non viene utilizzato con combustibili a lunga catena carboniosa come RP-1, per il problema legato ai suoi prodotti di combustione. Si preferisce usare questo ciclo con combustibili come idrogeno o metano (come nei nuovi motori sviluppati da SpaceX). I motori con cicli chiusi e OR, storicamente sviluppati dai russi dall'inizio degli anni '50, venivano alimentati anche con carburanti come il kerosene. 

Concludendo, si può dire che la scelta del ciclo di alimentazione per il motore F-1 è ricaduta su un ciclo a gas per una combinazione di fattori come: esigenze prestazionali, del singolo motore, elevate in termini di potenza e per quel tipo di combinazione di propellenti(con un trade-off sulla perdita di prestazione, come vedremo poi); limitazioni date dallo sviluppo degli altri cicli di alimentazione in quel periodo storico; minore complessità del sistema stesso.

\cfig{staged_comb}{Staged Combustion Cycle FR}{staged_comb}{0.3}
\cfig{full_cycle}{Staged Combustion Cycle Full Flow}{full_cycle}{0.3}

\subsection{Analisi delle variazioni di performance introdotte dal ciclo a gas}
Nella seguente sezione, verrà analizzato come l'introduzione del ciclo a gas ha delle ripercussioni sulle performance del sistema stesso, in particolare sull'impulso specifico del sistema motore, che verrà denotato in questo modo: $I_{s,oa}$, l'abbreviativo "oa" si riferisce al termine "overall". Importante è la distinzione tra l'impulso specifico del motore intero (appena introdotto) oppure della camera di spinta (più ugello), che sarebbe quello teorico denotato con $I_{s,tc}$.  

Per poter trattare teoricamente le limitazioni, si parte dal presupposto che a parità di altri fattori (come quota, mixture ratio, rapporto di espansione) un aumento di pressione in camera di combustione aumenta le prestazioni del sistema. Infatti, analizzando la spinta si vede che;
\begin{empheq}{gather}
T = \dot{m_p}u_e +  A_e \left(p_a - p_e\right) \qquad
u_e = u_e\left(p_c\right) = 
\end{empheq}
Si vede che aumentando la pressione in camera, a questo livello di approssimazione, la spinta aumenta indefinitamente per cui anche l'impulso specifico $I_{s,tc}$ aumenterà poichè esso è definito da:
\begin{empheq}{gather}
I_{s,tc} = \frac{T\left(p_c\right)}{\dot{m}_p} = u_e\left(p_c\right) +  \frac{A_e \left(p_a - p_e\right)}{\dot{m_p}} 
\end{empheq}
Questo ci permetterebbe di concludere che un aumento di pressione illimitato in camera di combustione, aumenta indefinitamente le prestazioni: sia perchè l'ugello può espandere a pressione più alta sia per un contributo statico. Introducendo il sistema di alimentazione di potenza in parallelo come il GG, si ha in realtà un calo di prestazioni rispetto all'andamento teorico. Questo fatto può essere compreso a livello qualitativo considerando che in un tale sistema, aumentare la pressione in camera significa aumentare le prestazioni delle pompe che richiedono quindi più potenza dalla turbina. A parità di salto di pressione in turbina e prodotti di combustione del GG, l'unico modo che si ha per aumentare la potenza prodotta dalla turbina è aumentare lo spillamento di portata dal flusso che andrà poi in camera. Questo flusso, oltre a non conseguire una combustione ottimale (ovvero con rapporto O/F molto diverso da quello della camera), viene espanso a velocità molto più basse e questo implica una perdita di prestazioni che può essere quantificata in questo modo: \cite{AIAA_book_1}
\begin{empheq}{gather}
I_{s,oa} = \left(1 - \frac{\dot{m}_{gg}}{\dot{m}_p}\right)I_{s,tc} + \frac{\dot{m}_{gg}}{\dot{m}_p}I_{s,gg}
\\
I_{s,gg} = I_{s,gg} \left(u_{e,gg}, P_{c,gg}, \dot{m}_{gg}\right)
\\
u_{e,gg} = u_{e,gg} \left(P_{c,gg}\right) = ...
\\
P_{c,gg} = 0.85 \cdot P_{c}
\\
\dot{m}_{gg} = \dot{m}_{gg}\left( Pwr, \eta_{t}, \epsilon , T_{in}, c_p \right) = \frac{Pwr}{\eta_t T_{in} c_{p,gg }\epsilon }
\\
Pwr = Pwr_{LOX} + Pwr_{RP-1} = \frac{\Delta P_{lox} \dot{m}_{lox}}{\eta_{P,lox} \rho_{lox}} + \frac{\Delta P_{rp1} \dot{m}_{rp1}}{\eta_{P,rp1} \rho_{rp1}}
\\
I_{s,tc} = I_{s,tc} \left(u_{e,tc}, P_c, \dot{m}_{ch}\right)
\\
u_{e,tc} = u_{e,tc} \left(P_{c}\right) = ...
\\
\dot{m}_{ch} = \dot{m}_p - \dot{m}_gg
\\
\end{empheq}
Si è ipotizzata, come best practice, che la pressione in camera del GG sia circa l' .. della pressione nella camera principale. Si ipotizza inoltre che il salto di pressione attraverso le pompe sia proporzionale, in prima approssimazione, alla pressione in camera con costante di proporzionalità ottenuta dalla divisione delle due quantità note: prevalenza pompa (sia LOX che RP-1) e pressione in camera nominale dell'F-1. \cite{AIAA_book_1}
Si sono implementate tali equazioni in Matlab, in un grande range di pressioni, per poter notare l'effetto di introduzione del gas generator nel sistema. 
\cfig{I_spec_see}{Confronto Impulso specifico a livello mare}{I_spec_see}{0.6}
Si nota come, al valore di pressione associato al motore F-1, ci sia un calo di prestazione. Il valore di impulso cala da 291 secondi a circa 285 secondi, il tutto calcolato a livello mare. Questo modello non considera perdite prestazionali del flusso 3D (con eventuali separazioni asimmetriche e considerazioni sulle onde d'urto), in questa situazione infatti l'ugello è in condizione sovra-espansa. Queste perdite sono invece successivamente considerate nella modellazione tramite il software RPA. 
Dai grafici precedenti si nota anche come l'ottimo prestazionale per l'impulso specifico del sistema complessivo sia a pressioni in camera molto elevate (attorno ai 30 MPa, ovvero nell'ordine di pressione di 300bar). Un valore di pressione così elevato richiede una revisione di tutti i sistemi di alimentazione, a partire dalla turbopompa. Tale valore di pressione si trova tuttora nei motori LRE di nuova generazione come il Raptor, che tra l'altro non possiedono il ciclo a gas.