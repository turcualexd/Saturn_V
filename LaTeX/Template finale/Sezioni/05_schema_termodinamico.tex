\section{Schema termodinamico}
\label{sec:schema termodinamico}

Lo schema termodinamico semplificato del sistema propulsivo F-1 viene presentato di seguito. Per poter trattare le principali grandezze termodinamiche quali sono la pressione P, la temperatura T e la portata massica $\dot{m}$, si è deciso di consultare i manuali del motore per poter estrapolare uno schema semplificativo a blocchi. \cite{engine_manual} Nello schema presentato viene introdotto il sistema a ciclo generatore di gas che permette l’alimentazione della turbopompa. Supponendo il funzionamento a regime (Main Stage), l’alimentazione è completamente auto sostenuta finchè non viene soppressa dai computer di bordo (al termine del $t_{burn}$) o viene esaurito il propellente. 

Qualitativamente, dai due serbatoi di LOX e RP1 viene spillata una portata, che viene trattata dalla turbopompa che porta in pressione i due liquidi. I due tank sono messi leggermente in pressione da un gas inerte: elio (e GOX nel tank LOX). Il motivo per cui si preferisce avere un gas in pressione è che permette una uscita facilitata dai due tank ed evita la cavitazione man mano che vengono svuotati i serbatoi. La turbopompa sarà trattata in dettaglio nei paragrafi successivi, data la sua complessità costruttiva. Per questo schema è sufficiente sapere che essa ha il compito di portare ad una certa pressione i due liquidi. Per poter alimentare le pompe, viene calettata sullo stesso asse in comune una turbina. Questa turbina viene mossa da dei gas caldi combusti in una piccola camera di combustione. Questo sottosistema viene chiamato Gas Generator o GG, viene alimentato da una portata spillata dopo le turbopompe della stessa coppia RP1/LOX con un eccesso di combustibile per evitare temperature elevate in ingresso turbina. I gas caldi in uscita dal GG vengono ulteriormente sfruttati per poter scaldare e quindi pressurizzare l’elio, successivamente tali gas di scarico vengono posti in uno tubo circonferenziale all’ugello nella posizione 10:1 di espansione delle aree del divergente, dove vengono scaricati sulla parete interna dell’estensione dell’ugello. Questo viene fatto per creare un film di gas relativamente freddi che hanno il compito di alleviare il carico termico sopportato da questa porzione di ugello (vedi appendice per rappresentazioni grafiche dettagliate). Il raffreddamento della parte superiore dell’ugello viene effettuato facendo passare il combustibile in diversi tubi esterni posti nella sezione tra gola e divergente 10:1, il combustibile dopo aver assorbito calore viene introdotto in camera di combustione. 

In tabella e figura vediamo alcuni dati rappresentativi del sistema intero. Si è ipotizzato di trattare i gas come gas perfetti, di assumere come dati alcuni rendimenti e alcune grandezze caratteristiche. 

