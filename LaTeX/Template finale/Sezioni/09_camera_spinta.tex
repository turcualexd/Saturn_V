\section{Camera di spinta}
\label{sec:camera spinta}

\subsection{Descrizione del sistema}
\label{subsec:descrizione camera spinta}

L’intero gruppo della camera di spinta è costituito dal LOX dome, dal piatto di iniezione e dal corpo della camera di spinta.

In questa sezione verrà analizzato dettagliatamente il corpo della camera di spinta: esso è composto da una camera di combustione per bruciare propellenti seguita da un ugello di espansione a forma di campana, necessario ad espellere a velocità elevata i gas prodotti dai propellenti bruciati e poter così generare la spinta.

La camera di spinta è di circa 3.35 metri di lunghezza e 2.74 metri di diametro all'estremità inferiore dell'ugello. Il suo corpo è formato da tubi fino al piano in cui il rapporto di espansione diventa pari a 10:1; in questi scorre il 70\% del carburante, fornendo un raffreddamento rigenerativo ed evitando che il materiale del tubo si sciolga durante il funzionamento del motore.

I tubi che compongono la camera di spinta sono costruiti in Inconel X-750, una lega a base di nichel resistente alle alte temperature e trattabile termicamente, e sono rinforzati strutturalmente da una serie di fasce attorno all'ugello.

Il corpo sopra il piano corrispondente a rapporto di espansione 3:1 (circa 76.2 cm sotto il piano di gola) è costituito da 178 tubi primari con diametro esterno di 2.78 cm, mentre dal piano con rapporto di espansione 3:1 al piano a 10:1 si biforcano diventando 356 tubi secondari con diametro esterno di 2.54 cm.

Questa biforcazione è dovuta principalmente alla geometria dell'ugello e alle proprietà fisiche del materiale usato. La circonferenza di un ugello è minima in corrispondenza della gola mentre aumenta nella sezione di espansione: l'entità dell'aumento di circonferenza ottenibile con un numero fisso di tubi è quindi limitata da quanto questi ultimi possano essere lavorati per rastremazione. Quando si raggiunge il punto in cui la circonferenza non può essere ulteriormente aumentata con il numero prefissato di tubi si rende quindi necessario un giunto di biforcazione.

L’Inconel X-750 è stato scelto come materiale poiché forniva gli elevati rapporti forza-peso necessari per resistere ai requisiti di spinta del motore; inoltre, l'elevata resistenza di questa lega ha consentito la progettazione di tubi con sezioni di parete più sottili, con conseguente diminuzione del peso. Il design più sottile per i tubi fornisce anche un adeguato raffreddamento della camera di spinta, con circa due terzi del flusso totale di carburante che transita attraverso i tubi. Esso finisce poi in un collettore posizionato all'estremità inferiore della camera, da cui viene poi drenato da 4 porte di drenaggio posizionate a 90 gradi l'una dall'altra.

I gas di scarico della turbina, dopo essere passati attraverso lo scambiatore di calore, vengono convogliati al collettore di scarico della turbina, la cui funzione è quella di raccogliere e distribuire uniformemente il gas di scarico tra le pareti dell’estensione dell'ugello, che altrimenti non sarebbe raffreddato.

Le piastre divisorie all'ingresso e le alette di flusso nell'area di uscita contribuiscono alla distribuzione uniforme dei gas di scarico nell'estensione dell'ugello. Il collettore di scarico è saldato ad uno scudo ignifugo nella parete esterna della camera di spinta.

\subsection{Camera di combusione}
\label{subsec:camera_di_combustione}

Nel motore F-1 è presente una camera di combustione cilindrica con una parte finale convergente che termina con la sezione di gola.
La camera di combustione funge da involucro che deve mantiene i propellenti per un periodo sufficiente a garantire la completa miscelazione e combustione. Il tempo di permanenza richiesto, o tempo di residenza, è una funzione di molti parametri, ovvero combinazione di propellenti, le condizioni di iniezione e la geometria del combustore (rapporto di contrazione, numero di Mach, livello di turbolenza).

Un parametro utile relativo al volume della camera e il tempo di residenza è la "lunghezza caratteristica", L*, ossia il volume della camera diviso per l’area di gola:

FORMULEEEEE


Il concetto di L* è molto più facile da visualizzare rispetto al più elusivo "tempo di residenza", espresso in piccole frazioni di secondo, infatti si tratta di un sostituto per determinare il tempo di permanenza nella camera dei propellenti.

Un altro parametro fondamentale per il calcolo del tempo di residenza è la velocità caratteristica c*.
Il valore c* aumenta con L* fino a un massimo asintotico, ma l’aumento di L* oltre un certo punto tende a diminuire le prestazioni complessive del motore a causa di quanto segue: un maggiore L* si traduce in maggiore volume e peso della camera di spinta, con conseguente aumento della superficie che ha bisogno di raffreddamento e aumento delle perdite termiche e dovute all’attrito.
Il metodo abituale per stabilire la L* di un nuovo progetto della camera di spinta si basa in gran parte sull'esperienza passata con propellenti e dimensioni del motore simili. 
Nel caso della coppia di LOX/RP-1 la L* è compresa tra i valori 1÷1.30 metri, e per motivi progettuali descritti in precedenza è stata fissata la misura di L* a 1 m.
Partendo da questo dato e dalla dimensione dell’area di gola del motore è possibile modellare la camera di combustione.

Invertendo la formula della L* è possibile ottenere il volume della camera di combustione Vc che comprende, oltre alla parte cilindrica, anche la parte convergente.

FORMULAAAAAA

È possibile anche calcolare la velocità caratteristica c* e il tempo di residenza attraverso le seguenti due formule:

FORMULEEEEEE

Per calcolare le dimensioni reali di entrambe le parti della camera di combustione, ovvero la parte cilindrica e quella del convergente, è necessario fissare due parametri al fine dello sviluppo del modello.
Si tratta del rateo di contrazione, ossia il rapporto tra la sezione della camera cilindrica e la sezione di gola, e l’angolo di inclinazione del convergente

FORMULEEEEEEE

Sfruttando la trigonometria si ottiene la lunghezza assiale del tratto convergente e considerando quest’ultimo come una figura tronco conica si ottiene il volume da rapportare a quello calcolato in precedenza, così da ottenere il valore percentuale dell’ingombro del convergente, utilizzato nei calcoli seguenti.
Si ottiene così una camera di combustione con le seguenti misure:

TABELLAAAAAAAA


\subsection{Modellazione dell'ugello}
\label{subsec:modellazione ugello}

L’obiettivo principale che si persegue nella progettazione dell’ugello di un endoreattore è quello di ottenere una forma che minimizzi le perdite di spinta per un qualsiasi rapporto di espansione richiesto.

Si procede quindi ad illustrare il metodo ideato da Rao per una progettazione ottimale rispetto ad un ugello di forma tronco-conica, con lo stesso rapporto di espansione, preso da riferimento:

\begin{empheq}{equation*}
L_{con} = \frac{\left( \sqrt{\epsilon} - 1 \right) - R_t}{\tan {15\degree}}
\end{empheq}
\vspace{5pt}

dove $ R_t $ indica il raggio di gola dell’ugello e 15° è l’angolo standard di semi-apertura dell’ugello.

\vspace{5mm}

\rfig{ugello_TOP}{Definizioni geometriche}{ugello_TOP}{0.5}

La forma a campana ottimale può essere approssimata da una parabola inclinata grazie a considerazioni geometriche, permettendo anche di abbozzare velocemente una forma dell’ugello che contempla una perdita di prestazioni trascurabile a livello di spinta. Proprio per questo motivo, questa tipologia è anche chiamata ugello TOP (Thrust Optimized Parabolic) ed ha effettivamente trovato applicazione pratica nei vettori di lancio perché ha performance migliori quando sovra-espande a livello del mare (le pareti dell’ugello TOP aiutano a ritardare la separazione del flusso grazie ad un’elevata contropressione) rispetto ad un ugello ottimizzato perfettamente a campana. Inoltre, la forma dell’ugello varia in modo minimale in base ai propellenti usati e perciò una stessa famiglia di ugelli TOP può essere adattata per qualsiasi combinazione di ossidante e combustibile.

I parametri di partenza sono: il rapporto di espansione $ \epsilon $, il raggio di gola $ R_t $ e la percentuale di campana $ \%_{bell} $ che si vuole ottenere; quest’ultimo valore deve essere compreso tra il valore massimo di 85\%, a cui si raggiunge un livello di efficienza dell’ugello del 99\% e che può essere aumentato solo di un ulteriore 0.2\% con una percentuale di campana 100\%, e il valore minimo del 70\%, a cui si comincia ad ottenere un notevole degrado di prestazioni. Si ricavano quindi a cascata:

\vspace{5pt}
\begin{empheq}{alignat*=2}
& R_e = \sqrt{\epsilon} \, R_t		&\qquad		& \text{raggio della sezione d’efflusso}
\\
& L_{ugello} = \%_{bell} \frac{\left( \sqrt{\epsilon} - 1 \right) - R_t}{\tan {15\degree}}
&\qquad		& \text{lunghezza dell’ugello}
\end{empheq}
\vspace{5pt}

Si ricavano poi gli angoli $ \theta_n $, riferito al punto di inflessione N, e $ \theta_e $, riferito alla sezione d’uscita, per interpolazione grafica da curve analitiche ottenute sperimentalmente per determinati valori di $ \%_{bell} $ (grafico in \autoref{fig:angoli_bell}).

La prima parte di modellazione vera e propria consiste nella costruzione della gola dell’ugello secondo una geometria ottimale usata da Rao (ai tempi ingegnere alla Rocketdyne) e basata sull’intersezione di due archi di circonferenza definiti come segue:

\begin{empheq}{equation*}
x = 1.5 \, R_t \cos \theta	\qquad	y = R_t \left( 1.5 \, \sin \theta + 1.5 + 1 \right)
\end{empheq}

per la sezione di entrata, con $ -103\degree < \theta < -90\degree $ (l’angolo iniziale di -103° è scelto dal progettatore della camera di combustione \cite{nozzle_design} ma può anche essere fissato ad un valore differente);
\vspace{5pt}

\begin{empheq}{equation*}
x = 0.382 \, R_t \cos \theta	\qquad	y = R_t \left( 0.382 \, \sin \theta + 0.382 + 1 \right)
\end{empheq}

per la sezione di uscita, con $ -90\degree < \theta < \theta_n - 90\degree $.
\vspace{5mm}

Per la costruzione della campana è invece necessario definire prima tre punti geometrici:

\begin{itemize}[wide,itemsep=8pt,topsep=8pt]

\item
punto di inflessione N: $ \quad \text{N} = \begin{bmatrix} N_x \\ N_y \end{bmatrix} = \begin{bmatrix}
0.382 \, R_t \cos \left( \theta_n - 90\degree \right) \\
R_t \left[ 0.382 \, \sin \left( \theta_n - 90\degree \right) + 0.382 + 1 \right]
\end{bmatrix} $
\item
punto tangente alla sezione d'efflusso E: $ \quad \text{E} = \begin{bmatrix} E_x \\ E_y \end{bmatrix} = \begin{bmatrix} R_e \\ L_{ugello} \end{bmatrix} $
\item
punto Q di intersezione delle rette passanti da N con inclinazione $ \theta_n $ e da E con inclinazione $ \theta_n $:

\begin{empheq}{alignat*=2}
&\overrightarrow{NQ} = m_1 x + C_1 \; \text{con} \; m_1 = \tan \theta_n \; \text{e} \; C_1 = N_y - m_1 N_x
&\qquad
& Q_x = \frac{C_2 - C_1}{m_1 - m_2}
\\
&\overrightarrow{QE} = m_2 x + C_2 \; \text{con} \; m_2 = \tan \theta_e \; \text{e} \; C_2 = E_y - m_2 E_x
&\qquad
& Q_y = \frac{m_1 C_2 - m_2 C_1}{m_1 - m_2}
\end{empheq}

\end{itemize}
\vspace{5pt}

La campana infine risulta essere una curva di Bézier quadratica di equazione:

\begin{empheq}{alignat*=2}
& x(t) = \left( 1 - t \right)^2 N_x + 2 \left( 1 - t \right) t \, Q_x + t^2 E_x &\qquad
& 0 \le t \le 1 \\
& y(t) = \left( 1 - t \right)^2 N_y + 2 \left( 1 - t \right) t \, Q_y + t^2 E_y &\qquad
& 0 \le t \le 1
\end{empheq}

\vspace{5pt}
\subsection{Cooling della camera di spinta}
\label{subsec:cooling camera}

I motori a propellente liquido sfruttano varie tecnologie per il raffreddamento delle pareti della camera di spinta. Nel caso analizzato, il motore F-1 sfrutta due tipi di raffreddamento: il film cooling, che protegge le pareti dell'estensione dell'ugello attraverso un sistema di iniezione, e il raffreddamento rigenerativo, che utilizza il combustibile come fluido refrigerante passante attraverso una serie di tubi che costituiscono la parete stessa dell'ugello.

\subsubsection{Scambio termico convettivo e film cooling}

Per poter analizzare la protezione termica delle pareti della camera di spinta, è in primo luogo necessario stimare il valore di scambio termico convettivo dai gas combusti alle pareti stesse.

La trattazione dello scambio termico convettivo nel caso preso in analisi viene affrontata tenendo conto dalle alte velocità dei gas combusti: ciò porta alla formazione di uno strato limite, che si assottiglia lungo il convergente in concomitanza con l'accelerazione del fluido subsonico, raggiungendo il minimo in gola, per poi ispessirsi nel divergente. Lo scambio termico è quindi un problema riguardante lo strato limite e il suo spessore, la sua temperatura e la velocità del fluido.IMMAGINE.

Poichè si raggiunge il minimo spessore dello strato limite in gola, ci si aspetta di avere il massimo scambio convettivo nel punto dell'ugello in cui il rapporto $A_t$ su A è minimo. Questa osservazione è di particolare rilevanza, poiché, come si vedrà più avanti, per determinare la portata massica necessaria per il film cooling, si utilizzerà l'area della sezione minima dell'estensione dell'ugello, che corrisponde all'area della sezione con rapporto di espansione 10:1.

Risulta complicato determinare il valore preciso del calore scambiato in modo convettivo tra i gas combusti e le pareti, in quanto lo strato limite è fortemente influenzato da vari fattori, quali la curvatura delle pareti, il gradiente di pressione in direzione assiale, il gradiente di temperatura associato all'alta intensità del flusso di calore; è tuttavia possibile utilizzare un metodo semi-empirico per farne una accurata stima.

Lo scambio convettivo per unità di area lato gas all’interfaccia tra fluido e superficie solida dipende da un coefficiente detto "coefficiente di film" $h_g$ : FORMULA CON \[ASTERISCO\] PERCHE RIPRESA ALLA FINE DELLA TRATTAZIONE
dove $T_{wg}$ è la temperatura della parete dal lato caldo, mentre $T_{aw}$ è la temperatura adiabatica della parete. Per determinare la temperatura $T_{wg}$ è sufficiente moltiplicare la temperatura in camera di combustione per un fattore pari a 0.8, fattore che tiene conto della presenza di depositi solidi di carbonio sulle pareti.

Per comprendere il significato della temperatura adiabatica $T_{aw}$ è necessaria una piccola digressione. La velocità del fluido all'esterno dello strato limite è la velocità del flusso libero e, attraversando lo strato limite perpendicolarmente alla parete, la velocità diminuisce fino ad annullarsi per soddisfare la condizione di aderenza. La temperatura a parete dovrebbe perciò essere pari alla temperatura di ristagno, ossia la temperatura raggiunta quando tutta l'energia cinetica viene trasformata in energia termica senza alcuna perdita. Nel caso di flussi molto veloci, l'aumento di temperatura è abbastanza elevato da provocare un processo di rallentamento viscoso non adiabatico. Per questo motivo, nell'ipotesi di parete adiabatica verso l'esterno, avviene un significativo scambio termico dal fluido in prossimità della parete, caratterizzato da bassa velocità e alta temperatura statica, verso il fluido più lontano dalla parete. A parete si avrà quindi una temperatura $T_{aw}$ più bassa della temperatura che caratterizza il flusso libero, mentre all'interno dello strato limite, affinché venga soddisfatta l'equazione dell'energia per flussi stazionari, deve essere necessariamente presente una regione in cui la temperatura è più alta di quella del flusso libero. Si delinea un andamento della temperatura come schematizzato in figura. IMMAGINE

Nel caso preso in esame, il valore della temperatura $T_{aw}$ è determinabile scalando la temperatura in camera di combustione di un fattore detto "recovery factor" $f_{aw}$ , definito come legame tra $T_{aw}$ e le temperature statica e totale del flusso libero e con valore compreso tra 0.9 e 0.98. In particolare il recovery factor rappresenta il rapporto tra l'aumento della temperatura causato dall'attrito e l'aumento causato dalla compressione adiabatica. Esso è determinabile sperimentalmente o può essere stimato in base ad una correlazione semplificata e applicata nel caso di flusso turbolento, basata sul numero di Prandtl: FORMULA (Pr elevato 0.33), dove il numero di Prandtl può essere approssimato come segue: FORMULA + VALORE DI MU.

È necessario precisare che la temperatura in camera di combustione $T_c$ utilizzata è quella teorica moltiplicata per il fattore correttivo della velocità caratteristica. Quest'ultima infatti dipende unicamente dalla variabile $T_c$, quindi il fattore correttivo per le due grandezze coincide. Il suddetto fattore varia in un intervallo compreso tra 0.87 e 1.03, mentre il valore utilizzato nella trattazione, ossia 0.975, è il valore sperimentale adottato dal libro Modern Engineering for Design of Liquid Rocket Propellant.

Per stabilire il valore di calore scambiato per unità di area rimane da calcolare solo il coefficiente di film $h_g$, che può essere ricavato mediante la seguente formula: FORMULA
dipendente, tra gli altri parametri, dal numero di Prandlt, dalla viscosità, dal raggio di curvatura in gola dell'ugello e dal fattore correttivo SIGMA, che tiene conto delle variazioni di proprietà attraverso lo strato limite. 
Il raggio di curvatura in gola dell'ugello è stato ricavato tramite approssimazione di Rao. 
L'unico valore incognito è quindi SIGMA: questo fattore può essere determinato in termini di temperatura di combustione, temperatura locale a parete e numero di mach locale mediante la relazione di Bartz; in alternativa è possibile determinarlo per interpolazione, in funzione del rapporto $T_{wg}$ DIVISO $T_c$ e del valore di GAMMA. IMMAGINE. Assumendo il rapporto $T_{wg}$ DIVISO $T_c$ pari a 0.8, ricavato sperimentalmente e adottato nella trattazione nel volume Liquid Rocket Engine e che tiene conto della presenza del deposito di carbonio sulle pareti, noto il valore di GAMMA, pari a 1.2439, e ricordando che il fine ultimo del calcolo è progettare il film cooling dell'ugello aggiuntivo (intervallo di rapporto di espansione 0.1 - 1.6), è possibile determinare dal grafico che il valore del fattore correttivo si attesta intorno a 0.7 in tutto l'intervallo in esame.
Il coefficiente di film dipende infine anche dal rapporto $A_f$ su A, dove A è l'area della sezione locale. Il valore di questo rapporto è stato fatto variare per via numerica tra 0.1 e 1.6, calcolando poi per ciascun valore il corrispondente coefficiente di film e, in seguito, la corrispondente portata minima in massa per effettuare un adeguato film cooling.
Il valore di $h_g$ così ottenuto tiene unicamente in conto del calore scambiato tra fluido e parete, senza considerare la presenza di eventuali prodotti di combustione allo stato solido. I prodotti di combustione della coppia LOX – RP-1 contengono circa lo 37 in percentuale di particolato solido. Queste particelle tendono a depositarsi sulle pareti della camera di combustione, formando un efficace strato isolante: la valutazione quantitativa dell’efficacia dell’isolamento di questo strato, necessaria per il corretto calcolo dello scambio di calore, può essere effettuata solo sperimentalmente. Lo strato isolante è formato a sua volta da uno strato superficiale di fuliggine, che ne sovrasta uno piu tenace: quest’ultimo aumenta la resistenza termica lato gas, tale che la temperatura del deposito di carbonio all’interfaccia lato gas si avvicini alla temperatura del gas all’aumentare dello spessore del layer di carbonio.
Per il calcolo dello scambio termico nel caso di presenza di deposito solido sulle pareti della camera, l’equazione [ASTERISCO] viene corretta dalla seguente equazione, che vede una sostituzione del coefficiente di film con il coefficiente di conduttanza termica complessiva lato gas $h_{gc}$ FORMULA
Questo coefficiente considera sia $h_g$ sia il coefficiente di resistenza causata dal deposito solido $R_d$ , il cui valore è dipendente dal rapporto di espansione e dalle condizioni di pressione e rapporto di miscela FORMULA
Dopo aver calcolato tutti i parametri necessari, è quindi possibile progettare il sistema di film cooling dell'estensione dell'ugello. Il film cooling delle pareti interne è ottenuto iniettando i gas di scarico della turbina, forniti alla cavità tra le pareti dal collettore di scarico della turbina, nel flusso di scarico della camera di spinta attraverso fessure formate da 23 file di scandole sovrapposte che formano la parete interna.
Per lo sviluppo dei calcoli si consideri che il fluido di lavoro è gas con presenza di particolato, ed è quindi possibile utilizzare la relazione di Hatch e Papell, sostituendo al coefficiente $h_g$ il coefficiente $h_{gc}$ appena calcolato FORMULA
ove $T_{co}$ è la temperatura iniziale del fluido refrigerante, ossia la temperatura all'uscita dello scambiatore; $C-{pvc}$ è il calore specifico medio a pressione costante del fluido refrigerante, che è stato numericamente ottenuto interpolando i valori dopo la turbina in frozen equilibrium; infine $NU_{c}$ è l'efficienza del film cooling ed è un fattore che ha scopo correttivo, ossia tiene conto della quantità di refrigerante gassoso perso nel flusso di gas di combustione che quindi non produce effetti di raffreddamento. I valori dell'efficienza variano dal 25 al 65 in percentuale in funzione della geometria dell'iniezione del refrigerante e dalle condizioni di flusso. 
Dalla precedente equazione si evince che l'apporto termico dipende dal coefficiente di scambio $h_gc$ e dalla differenza tra temperatura adiabatica a parete e la temperatura del refrigerante; il calore assorbito è proporzionale alla capacità termica del film refrigerante dal valore di temperatura iniziale a quello finale. Esiste quindi un equilibrio tra apporto di calore e aumento di temperatura del refrigerante: raggiunto questo equilibrio si raggiunge la condizione adiabatica e la superficie della parete avrà localmente la medesima temperatura del film; infatti la temperatura della parete varierà assialmente dalla temperatura iniziale del refrigerante fino alla temperatura massima ammissibile.
L'obiettivo del calcolo è perciò quello di determinare la portata massica di fluido refrigerante per unità di area $G_c$, che poi verrà moltiplicato per l'area dell'estensione dell'ugello ad ottenere il valore di portata massica necessaria per il film cooling. Si noti che la portata dipende dal valore $h_{gc}$ , a sua volta dipendente dal rapporto $A_t/A$, che è stato fatto variare tra 1:10 e 1:16 : la portata massica che sarà sufficiente a raggiungere un efficiente film cooling in ogni sezione dell'ugello sarà la portata massima tra le portate calcolate, ossia quella ottenuta per rapporto $A_t$ su A maggiore e perciò A minore, quindi l'area della sezione 1:10. Il valore di Gc ottenuto è minore della portata elaborata dal gas generator, e questo è un risultato prevedibile in quanto il valore di portata passante per il gas generator è dettato dai requisiti di potenza della turbina e non dalle esigenze del film cooling. È stato perciò dimostrato che la portata massica elaborata è sufficiente a raggiungere l'obiettivo desiderato di raffreddamento delle pareti.

\subsubsection{Regenerative cooling}
\label{subsubsec:regenerative cooling}

Il motore preso in esame sfrutta lo scambio termico rigenerativo come tecnica di raffreddamento delle pareti della camera di spinta, in particolare dalla gola e per la lunghezza dell'ugello fino al piano caratterizzato da rapporto di espansione 10:1. Il regenerative cooling utilizza una quota parte di combustibile stivato, circa il 70 per cento, come refrigerante: esso viene indirizzato in una serie di tubi opportunamente sagomati saldobrasati insieme che costituiscono la parete stessa dell'ugello di efflusso. Lo scambio di calore avviene quindi tra due flussi in movimento separati da una parete.
Questa tecnica vanta di alcuni importanti vantaggi, tra i quali il fatto che non comporti nessuna perdita di prestazioni, infatti l'energia termica assorbita del refrigerante viene restituita all'iniettore, e abbia una struttura relativamente leggera. Tuttavia si possono riscontrare alcuni svantaggi, come alte perdite di pressione per elevati livelli di flusso di calore.

La seguente figura IMMAGINE descrive la variazione di temperatura durante lo scambio di calore per regenerative cooling: a sinistra scorrono i gas combusti a contatto con il boundary layer e la cui temperatura è $T_{aw}$ (il cui significato è illustrato nel sottoparagrafo precedente), ossia la temperatura che verrebbe raggiunta dalla parete nel caso di parete adiabatica o isolante, che diminuisce sensibilmente all'interno del boundary layer fino a raggiungere la temperatura della parete lato gas $T_{wg}$. All'interno dello spessore della parete la temperatura continua a diminuire raggiungendo la temperatura $T_{wc}$, ossia la temperatura della parete a contatto con il refrigerante; quest'ultimo quindi sarà caratterizzato dalla temperatura $T_{c0}$ (bulk temperature del refrigerante).

Proprio a causa dello scambio di calore tra gas e refrigerante, la temperatura $T_{c0}$ aumenterà dal punto di ingresso fino al momento in cui l'RP-1 lascerà il condotto di raffreddamento: essa è quindi una funzione del calore assorbito e della portata. A livello strutturale è necessario svolgere il dimensionamento nel punto più critico, ossia nel tubo di ritorno all'altezza della gola, ossia l'ultima sezione attraversata dal refrigerante prima di essere immesso nella camera di spinta.

L'obiettivo ultimo del regenerative cooling è quello di mantenere la temperatura della parete al di sotto della temperatura critica alla quale possono realizzarsi fusioni localizzate o un decremento delle prestazioni del materiale. La temperatura limite nel caso della parete della camera di spinta dell'F-1, realizzata in Inconel X750, è tra 1100 K e 1255 K.
Identificate le temperature caratteristiche del processo di raffreddamento è quindi possibile calcolare il flusso di calore come: FORMULA

Rimaneggiando la formula essa può essere riscritta in funzione del coefficiente globale di scambio termico FORMULE
dove $h_{gc}$ è la conduttività termica complessiva lato gas, $h_{c}$ è il coefficiente di scambio termico lato refrigerante, mentre t è lo spessore della parete e k la conduttività termica della parete della camera. Osservando la precedente equazione è possibile introdurre un ulteriore requisito che il regenerative cooling deve soddisfare: per mantenere la temperatura della parete entro valori contenuti, è necessario che la conduttività termica complessiva lato gas $h_{gc}$ sia minimizzata, mentre il coefficiente di scambio termico del refrigerante sia molto alto, cosi come il rapporto t DIVISO k. Dal momento che la differenza di temperatura è inversamente proporzionale al coefficiente di scambio termico del flusso di calore, la diminuzione della temperatura sarà più rapida tra gas caldo e parete interna della camera.

Se per determinare il valore di $h_{gc}$ è sufficiente ripercorrere la trattazione riguardante lo scambio termico convettivo, per comprendere il significato del coefficiente $h_c$ e determinarne il valore numerico è necessario approfondire il suo legame con pressione e temperatura critica del refrigerante.
IMMAGINE Verranno analizzati due possibili scenari, rappresentati in figura: la curva $A_i$ descrive l'andamento del legame temperatura della parete – flusso di calore nel caso di pressione minore della pressione critica mentre la curva $B_i$ rappresenta l'andamento del legame nella condizione di pressione maggiore della pressione critica.

Studiando la curva A, il tratto A1-A2 rappresenta lo scambio di calore nelle condizioni in cui la temperatura della parete lato coolant non ha ancora raggiunto la temperatura di saturazione, in corrispondenza della pressione del refrigerante. Alla temperatura del punto A2, superata la temperatura di saturazione, il combustibile inizia a bollire, creando quindi delle “bolle” nella fascia a ridosso della parete. Queste crescono di dimensione nel flusso liquido piu freddo fino a che la velocità di condensazione del vapore supera la velocità di vaporizzazione: le bolle iniziano a collassare. Questo processo, che avviene ad alta frequenza, è detto “Nucleate boiling” (ebollizione nucleata). In corrispondenza di questo fenomeno il coefficiente di scambio termico aumenta, causando un aumento contenuto della temperatura a parete per un'ampia gamma di flussi di calore. Lo scambio di calore caratterizzato da ebollizione nucleata è rappresentato dal tratto A2-A3. Alla temperatura corrispondente al punto A3, un ulteriore aumento del flusso di calore porta ad un incremento di concentrazione di bolle tale per cui esse si combinano in un film di vapore a cui consegue una forte diminuzione del coefficiente del trasferimento del calore (tratto A3-A4). Lo scambio di calore raggiunto al punto A3 definisce il limite superiore del nucleate boiling, valore che viene quindi utilizzato come limite di progetto per il sistema di raffreddamento rigenerativo.
La curva B descrive le varie fasi del legame flusso di calore- temperatura della parete nel caso in cui la pressione sia al di sopra di quella critica: in queste condizioni di il fenomeno di nucleate boiling non si manifesta. Queste condizioni portano ad un aumento di temperatura proporzionale all'incremento del flusso di calore: in questo modo si raggiunge la temperatura limite per un valore di scambio di calore minore. Per questo motivo si predilige una pressione che sia tra il 30 e il 70 in percentuale della pressione critica.
Il dimensionamento del sistema di regenerative cooling è finalizzato a stabilire il numero di tubi che compongono la parete dell'ugello d'efflusso e le dimensioni dei singoli tubi, in particolare il diametro interno e lo spessore. Prima di procedere alla trattazione matematica è necessario chiarire alcune assunzioni considerate durante lo svolgimento dei calcoli. Il dimensionamento viene effettuato nella condizione piu critica, ossia vengono dimensionati i tubi di ritorno nella sezione di gola, perché la gola rappresenta il punto caratterizzato dal maggior valore di flusso termico attraverso e attraverso la sezione finale dei tubi di ritorno scorre il refrigerante alla sua temperatura massima raggiunta dopo aver percorso tutto il sistema di raffreddamento; la forma dell'ugello d'efflusso fa fede alla modellazione illustrata precedentemente e viene perciò considerata nota: dalla modellazione e dalla simulazione RPA verranno ricavati il diametro di gola e i raggi di curvatura utili a determinare il raggio di curvatura medio R; il numero di tubi rimane costante fino al piano caratterizzato dal rapporto di espansione 3:1, per poi raddoppiare fino al piano con rapporto di espansione 10:1. Infine, avendo come variabili sia il numero di tubi sia il loro spessore, è necessario ipotizzare o fissare uno dei due dati: è stato quindi fissato il numero reale di tubi che compongono l'ugello nel primo tratto, ossia 178, mantenendo come incognita lo spessore.
I calcoli preliminari al dimensionamento permettono di determinare, tramite una trattazione analoga a quella illustrata per lo scambio convettivo, il valore di flusso di calore specifico q, funzione della conduttività termica, della temperatura adiabatica a parete e della temperatura della parete lato gas. 
La temperatura a parete lato gas $T_{wg}$ è determinata sperimentalmente (Modern Engineering), mentre la temperatura adiabatica a parete $T_{aw}$ è ottenuta moltiplicando la temperatura in camera di combustione $T_c$ per il fattore di recupero dello strato limite turbolento in gola (valore intermedio tra 0.9 e 0.98). Noto quindi il rapporto $T_{wg}$ DIVISO $T_c$ e il gamma dei gas combusti è possibile determinare il valore del fattore di correzione in gola sigma dai grafici riportati nella sezione precedente. Infine è possibile calcolare il coefficiente di scambio termico lato gas tramite la formula: FORMULA
e quindi il valore della conduttività termica lato gas FORMULA
IMMAGINE A LATO Dalle precedenti formule è possibile definire $R_d$ la resistenza termica causata dal deposito solido in gola, R il raggio di curvatura dell'ugello calcolato come media dei due raggi di curvatura $R_1$ ed $R_n$, $D_t$ il diametro di gola calcolato come due volte $R_t$.
Terminati i calcoli preliminari è possibile passare al dimensionamento vero e proprio del sistema di refrigerazione. Sono noti i valori relativi alla lega X750 (conducibilità termica $k_lega$, modulo di elasticità E, coefficiente di espansione termica a, coefficiente di Poisson v) e vengono assunti i valori di bulk temperature del combustibile in gola, la sua conducibilità termica $k_fuel$, la sua densità (di stivaggio) e una costante $C_1$ propria dell’RP1 utile per il calcolo del numero di Nusselt (valore adottato dalla trattazione del volume Modern Engineering)
A questo punto per determinare lo spessore t dei tubi è possibile implementare un ciclo for che permetta di calcolare il numero dei tubi al variare dello spessore, per poi interrompere il ciclo quando il numero eguaglia il numero di tubi imposto: in questo modo si ottiene il valore dello spessore necessario. I calcoli svolti si basano su considerazioni fisiche e su formule empiriche.
Il ciclo inizia con il calcolo della temperatura della parete lato combustibile FORMULA necessario per determinare il valore del coefficiente  di scambio termico del combustibile FORMULA.
Per regioni di temperatura subcritica caratterizzate dall'assenza di nucleate boiling, la relazione tra temperatura a parete e flusso di calore, che dipende per l'appunto dal coefficiente di scambio termico $h_c$, può essere descritto tramite l'equazione di Sieder-Tate per il trasferimento di calore turbolento ai liquidi che fluiscono nei canali: FORMULA
dove mu è la viscosità del combustibile alla temperatura $T_{co}$, mentre $mu_w$ è la viscosità del combustibile alla temperatura della parete in gola. Questa relazione può essere riscritta esplicitando i singoli termini: FORMULA \[1 ASTERISCO\]
ove le incognite sono il diametro dei tubi d e la velocità media del combustibile $V_co$. Quest'ultima può essere calcolata in funzione del diametro dei tubi e del loro numero FORMULA
con $W_f$ la portata massica di combustibile, corrispondente al 70 per cento della portata totale del combustibile.
Tramite una formula empirica è possibile esplicitare l'espressione che permette di definire il numero di tubi FORMULA \[2 ASTERISCHI\].
Il fattore 0.8 ha il ruolo di fattore di correzione: il centro dei tubi è collocato su una circonferenza, piuttosto che su una retta.
Sostituendo quindi $V_{co}$ all'interno dell'equazione (1 ASTERISCO) esplicitandone N ed eguagliando l'espressione trovata all'equazione (2 ASTERISCHI) è possibile determinare il valore del diametro dei tubi. Sostituendo infine il valore trovato all'interno dell'equazione (2 ASTERISCHI) il si ottiene il numero dei tubi. Analizzando in un ciclo for i passaggi appena visti, è possibile determinare il valore di t tale per cui si ha il numero di tubi N desiderato.

\subsection{Confronto tra ugello 10:1 e 16:1}
\label{subsec:confronto ugello}

I motori F-1 prodotti dalla Rocketdyne avevano in origine un ugello il cui rapporto di espansione era 10:1; essi infatti non furono inizialmente progettati nello specifico per lo stadio di lancio del Saturn V. Gli ingegneri decisero quindi a posteriori di aggiungere un'espansione dell'ugello iniziale allo scopo di migliorare vari parametri del lanciatore: i più rilevanti sono l'impulso specifico nel vuoto e la quota a cui è raggiungibile l'espansione ottima (adattata alla traiettoria che il lanciatore avrebbe percorso).

Tale espansione non poteva essere raffreddata dal già presente regenerative cooling: si optò dunque per una soluzione che prevedesse l'utilizzo dei gas di scarico della turbina, ricchi di carbonio e quindi con bassa conducibilità termica, per il raffreddamento attraverso film cooling. Tale tipo di raffreddamento è realizzato immettendo i gas di scarico sulle pareti dell'ugello attraverso un collettore che ne abbraccia l'intera circonferenza.

L'estensione dell'ugello è realizzata da due pareti in lega di nickel intervallate da bande circolari in CRES: tale costruzione saldata conferisce ottima resistenza termica alle pareti e una buona resistenza agli sforzi radiali a cui l'ugello è sottoposto.

Di seguito sono confrontati i principali parametri del motore con e senza l'espansione dell'ugello, ricavati tramite il software RPA:

\begin{table}[H]

\centering
\begin{tabular}{|c|c|c|c|c|c|c|}
\hline
& $\bm{p_e \, [bar]}$ & $\bm{T_e \, [K]}$ & $\bm{H \, [kJ/kg]}$ & $\bm{\gamma}$ & $\bm{\rho \, [kg/m^3]}$ & $\bm{v_e \, [m/s]}$ \\
\hline
\textbf{10:1} & $0.803$ & $1673.7$ & $-5058.1$ & $1.2439$ & $0.1304$ & $2910.6$ \\
\hline
\textbf{16:1} & $0.423$ & $1473.0$ & $-5429.9$ & $1.2521$ & $0.0781$ & $3035.6$ \\
\hline
\end{tabular}

\vspace{5pt}

\begin{tabular}{|c|c|c|c|c|c|c|}
\hline
& $\bm{I_{vac} \, [s]}$ & $\bm{I_{opt} \, [s]}$ & $\bm{I_{sl} \, [s]}$ & $\bm{T_{vac} \, [kN]}$ & $\bm{T_{opt} \, [kN]}$ & $\bm{T_{sl} \, [kN]}$ \\
\hline
\textbf{10:1} & $297.36$ & $276.44$ & $270.95$ & $7816.3$ & $7266.3$ & $7122.2$ \\
\hline
\textbf{16:1} & $306.25$ & $288.59$ & $263.99$ & $8050.0$ & $7585.9$ & $6939.3$ \\
\hline
\end{tabular}

\caption{Confronto tra ugello 10:1 e 16:1}
\label{table:confronto_ugello}

\end{table}

Si può notare un miglioramento nella spinta e nell'impulso specifico in corrispondenza dell'espansione ottima e dell'espansione nel vuoto, mentre si ha un calo di prestazione a livello del mare: ciò è dovuto al fatto che l'ugello sovraespande in maniera più marcata a pressione standard, poiché il punto di espansione ottima viene spostato a pressioni inferiori. Ciò non risulta essere un problema in quanto la spinta rimane sufficiente al lancio, mentre i benefici ottenuti alle quote di missione sono rilevanti.