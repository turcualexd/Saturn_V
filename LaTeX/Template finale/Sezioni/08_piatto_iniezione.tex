\section{Piatto d'iniezione}
\label{sec:piatto iniezione}

\rfig{iniettore}{Piatto di iniezione}{iniettore}{0.4}

Gli iniettori sono collocati all’estremo superiore della camera di spinta e hanno lo scopo di distribuire il propellente in camera, regolando il rapporto di diluizione, la pressione e lo schema di spruzzo al fine di avviare e sostenere una combustione stabile. Per determinare questi valori sono stati necessari circa 3200 test su larga scala: al fine di generare un’esplosione controllata, risulta fondamentale che essa sia dinamicamente stabile, ossia che sia prevedibile e non crei punti caldi che porterebbero alla fusione di componenti del motore.

Il piatto di iniezione ha un diametro di 111.76 cm, è realizzato in CRES, acciaio molto resistente alla corrosione, ed è strutturato in 31 anelli, questi divisi in 13 scompartimenti da 2 deflettori circolari e 12 radiali. I vari compartimenti sono numerati da 1 a 13, mentre i deflettori sono identificati da lettere dalla A alla N.

La faccia del piatto di iniezione conta 1428 orifizi per l’ossidante e 1404 orifizi per il carburante. I getti vengono atomizzati attraverso una disposizione a doppietti omogenei, i vapori di combustibile e di ossidante si miscelano e reagiscono a formare i gas propellenti, destinati successivamente all’espansione in ugello.

Le 31 scanalature che costituiscono gli anelli consistono in 16 scanalature per il combustibile alternate alle 15 scanalature per l’ossigeno liquido. Gli anelli per il carburante sono alimentati attraverso un collettore radiale, mentre gli anelli per l’ossidante sono alimentati dal LOX dome tramite fori assiali.

Il LOX dome è considerabile il primo componente della camera di spinta: esso ha dimensioni 162.6 x 48.3 x 111.8 cm, con un peso di 818.3 kg; è realizzato in lega di ferro, rame e alluminio, con rivestimento in nichel e coating in silice. Il corpo del LOX dome contiene la flangia di attacco e i montanti di supporto per interfacciarsi con l’iniettore.

Il collettore invece incorpora due ingressi per il montaggio delle valvole di ossidante e una flangia per la linea di alimentazione dell’ossidante allo scambiatore di calore. Per evitare vorticità nell’ossidante, il collettore è isolato in due compartimenti da due argini toroidali. Solamente il 30\% del combustibile viene indirizzato direttamente al collettore, mentre il restante 70\% viene utilizzato per il raffreddamento rigenerativo della camera di spinta.

Sono inoltre presenti due alloggiamenti per gli ignitori del combustibile in ciascuno dei 12 scomparti esterni, e un alloggiamento del combustibile nel compartimento centrale, tutti collegati al collettore da singoli tubi di alimentazione.

Come detto precedentemente, è fondamentale ottenere una combustione stabile per non incorrere in danni alla camera di combustione: la stabilità è raggiunta principalmente mediante l’uso dei deflettori (baffles), oltre che variando l’angolo di impingement e il diametro degli orifizi in funzione della posizione sul piatto d’iniezione.

I deflettori in particolare alterano le caratteristiche acustiche di risonanza della camera di combustione, smorzando così le onde d’urto generate dalla combustione. I 12 deflettori radiali in rame sono alimentati dal deflettore circolare esterno. La configurazione dei deflettori utilizzata per il propulsore è stata ottenuta a seguito di vari test, nel quale si è ricercata la maggior stabilità di combustione possibile. Nella configurazione finale, i deflettori misurano circa 8 cm ciascuno e sono tutti dump-cooled, ovvero il raffreddamento è realizzato attraverso la circolazione di carburante all’interno del deflettore che viene successivamente scaricato nella camera di combustione. \cite{f-1_manual} \cite{JPP}

\begin{table}[H]

\centering
\begin{tabular}{|c|c|c|c|c|c|}
\hline
& $\bm{\dot{m} \, [kg/s]}$ & $\bm{A \, [m^2]}$ & $\bm{\Delta p \, [bar]}$ & $\bm{\rho \, [kg/m^3]}$ & $\bm{v \, [m/s]}$ \\
\hline
$\bm{fuel}$ & $742.09$ & $0.05484$ & $641$ & $810$ & $17.07$ \\
\hline
$\bm{oxidizer}$ & $1788.97$ & $0.03968$ & $2100$ & $1141$ & $40.54$ \\
\hline
\end{tabular}

\caption{Dati reali del piatto d'iniezione \cite{f-1_manual} \cite{JPP}}
\label{table:piatto iniezione}

\end{table}

A partire dai dati in \autoref{table:piatto iniezione}, è possibile inoltre stimare la velocità media teorica dei due propellenti all'uscita dai rispettivi iniettori:

\begin{empheq}{gather*}
	C_{D,f} = \frac{\dot{m}_{f}}{A_{f} \sqrt{2 \, \Delta p_{f} \, \rho_{f}}} = 13.28
	\qquad
	C_{D,ox} = \frac{\dot{m}_{ox}}{A_{ox} \sqrt{2 \, \Delta p_{ox} \, \rho_{ox}}} = 20.59
	\\
	v_{f} = C_{D,f} \sqrt{\frac{2 \, \Delta p_{f}}{\rho_{f}}} = 16.71 \, \text{m/s}
	\qquad
	v_{ox} = C_{D,ox} \sqrt{\frac{2 \, \Delta p_{ox}}{\rho_{ox}}} = 39.51 \, \text{m/s}
\end{empheq}

Tali risultati sono paragonabili alle velocità reali: la $v_{f}$ si discosta del 2.11\% dal valore reale, mentre la $v_{ox}$ si discosta del 2.54\%.
