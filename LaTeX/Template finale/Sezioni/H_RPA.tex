\section{RPA: definizione problema e funzionamento}
\label{appendix:rpa}
RPA è un software che permette di prevedere le prestazioni per motori a razzo. Il metodo di calcolo si basa su un approccio di minimizzazione dell'energia libera di Gibbs al fine di ottenere la composizione dei prodotti di combustione, l'analisi dell'ugello del flusso con shifting o frozen equilibrium e il calcolo delle prestazioni del motore, per camera di combustione finita o infinita.

RPA utilizza una libreria di specie chimiche espandibile basata sul database termodinamico NASA Glenn, che include dati per numerosi combustibili e ossidanti. È inoltre possibile definire nuovi componenti del propellente o importare componenti da database di specie PROPEP o CEA2.
Fornendo parametri propri del motore in analisi, come la pressione della camera di combustione, i componenti del propellente utilizzati e i parametri dell'ugello, il programma ottiene la composizione di equilibrio chimico dei prodotti di combustione, determina le sue proprietà termodinamiche e predice le prestazioni teoriche del razzo. 



