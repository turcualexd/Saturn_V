\section{Propellenti}
\label{sec:propellenti}

\subsection{Coppia di propellenti: RP-1 / LOX}
\label{subsec:coppia_propellenti}

Lo stadio S-IC utilizza la coppia di propellenti ossigeno liquido (LOX) e cherosene (RP-1), ovvero una coppia semi-criogenica. Questa combinazione offre un buon equilibrio tra efficienza e semplicità.

L'ossigeno liquido è un ossidante che reagisce facilmente con i combustibili, come il cherosene e l'idrogeno liquido, per produrre una combustione ad alta temperatura e alta pressione: difatti può produrre una maggiore spinta per unità di massa di propellente rispetto ad altri ossidanti.
La temperatura di ebollizione molto bassa (-182.96°C) lo rende un propellente criogenico, il che significa che deve essere mantenuto ad una temperatura uguale o minore rispetto a quella di ebollizione durante lo stoccaggio e l'utilizzo.
Tuttavia, a causa delle sue proprietà criogeniche, richiede una conservazione e una manipolazione estremamente accurate e può essere pericoloso se non gestito correttamente.

Il cherosene (RP-1), d'altra parte, è un combustibile ad alta densità energetica che brucia in modo pulito e consente un impulso specifico elevato rispetto ad altri combustibili idrocarburici, oltre ad essere facilmente reperibile e relativamente economico.
L'RP-1 è un tipo di cherosene raffinato, ovvero una miscela liquida di idrocarburi, che viene prodotto mediante il raffinamento del petrolio greggio che permette di rimuovere le impurità e migliorarne la stabilità e la consistenza. Il prodotto finito ha un alto contenuto di idrocarburi a catena lunga, che lo rende un combustibile altamente efficiente.

La scelta di questi propellenti ha preso in considerazione anche fattori come la sicurezza, l'affidabilità e la facilità di gestione.

\begin{table}[H]

\centering
\begin{tabular}{|c|c|c|c|c|}
\hline
& $\bm{\rho \, [kg/m^3]}$ & $\bm{T_{ebollizione} \, [K]}$ & $\bm{T_{congelamento} \, [K]}$ & $\bm{M \, [g/mol]}$ \\
\hline
\textbf{RP-1} & $810$ & $460 / 540$ & $225$ & $175$ \\
\hline
\textbf{LOX} & $1141$ & $90.2$ & $50.5$ & $32$ \\
\hline
\end{tabular}

\vspace{5pt}

\begin{tabular}{|c|c|c|c|c|c|c|}
\hline
& $\bm{I_{sp} \, [s]}$ & $\bm{O/F_{opt}}$ & $\bm{T_{comb} \, [K]}$ & $\bm{\gamma}$ & $\bm{\rho_{prod} \, [kg/m^3]}$ & $\bm{M_{prod} \, [g/mol]}$ \\
\hline
\textbf{RP-1/LOX} & $265$ & $2.56$ & $3670$ & $1.24$ & $1020$ & $21.9$ \\
\hline
\end{tabular}

\caption{Dati per la coppia di propellenti RP-1 / LOX}
\label{table:dati_propellenti}

\end{table}

\subsection{Propellente ipergolico}
\label{subsec:propellente_ipergolico}

Oltre ai due propellenti utilizzati per il funzionamento del motore è stato necessario l’utilizzo di un propellente ipergolico, ovvero un fluido ausiliario.
L'accensione del fluido ausiliario è un metodo per cui un liquido o un gas ipergolico, oltre al combustibile normale e all'ossidante, viene iniettato nella camera di combustione per un breve periodo durante l'operazione di avviamento del motore. Questo fluido produce una combustione spontanea a contatto con il combustibile o con l'ossidante.
Nel caso dell’F-1 sono stati utilizzati durante il processo di accensione Trietilborano (TEB - 85\%) e trietilalluminio (TEA - 15\%), che messi a contatto con l’ossigeno liquido sono in grado di avviare la combustione istantaneamente.