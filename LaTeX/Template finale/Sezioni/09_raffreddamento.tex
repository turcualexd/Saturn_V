\section{Sistemi di raffreddamento}
\label{sec:raffreddamento}

\subsection{Cooling della camera di spinta}
\label{subsec:cooling camera}

I motori a propellente liquido sfruttano varie tecnologie per il raffreddamento delle pareti della camera di spinta. Nel caso analizzato, il motore F-1 sfrutta due tipi di raffreddamento: il film cooling, che protegge le pareti dell'estensione dell'ugello attraverso un sistema di iniezione, e il raffreddamento rigenerativo, che utilizza il combustibile come fluido refrigerante passante attraverso una serie di tubi che costituiscono la parete stessa dell'ugello.

\subsubsection{Scambio termico convettivo e film cooling}
\label{subsubsec:film cooling}

\rfig{boundary_layer}{Strato limite sulla parete dell'ugello}{boundary_layer}{0.4}

Per poter analizzare la protezione termica delle pareti della camera di spinta, è in primo luogo necessario stimare il valore di scambio termico convettivo dai gas combusti alle pareti stesse. La trattazione dello scambio termico convettivo nel caso preso in analisi viene affrontata tenendo conto dalle alte velocità dei gas combusti: ciò porta alla formazione di uno strato limite, che si assottiglia lungo il convergente in concomitanza con l'accelerazione del fluido subsonico, raggiungendo il minimo in gola, per poi ispessirsi nel divergente. Lo scambio termico è quindi un problema riguardante lo strato limite e il suo spessore, la sua temperatura e la velocità del fluido. Poichè si raggiunge il minimo spessore dello strato limite in gola, ci si aspetta di avere il massimo scambio convettivo nel punto dell'ugello in cui il rapporto $A_t/A$ è minimo. Questa osservazione è di particolare rilevanza poiché per determinare la portata massica necessaria per il film cooling si utilizzerà l'area della sezione minima dell'estensione dell'ugello, che corrisponde all'area della sezione con rapporto di espansione 1:10.

Risulta complicato determinare il valore preciso del calore scambiato in modo convettivo tra i gas combusti e le pareti, in quanto lo strato limite è fortemente influenzato da vari fattori, quali la curvatura delle pareti, il gradiente di pressione in direzione assiale, il gradiente di temperatura associato all'alta intensità del flusso di calore; è tuttavia possibile utilizzare un metodo semi-empirico per fare una stima accurata.

Lo scambio convettivo per unità di area lato gas,all’interfaccia tra fluido e superficie solida, è descritto dal coefficiente di film $h_g$:
\vspace{2pt}
\begin{empheq}{equation*}
q = h_g \left( T_{aw} - T_{wg} \right)
\end{empheq}


Per comprendere il significato delle temperature  $T_{aw}$ e $T_{wg}$ è necessaria una piccola digressione trattata in \autoref{appendix:cooling_temp_definitions}.

Per stabilire il valore di calore scambiato per unità di area rimane da calcolare solo il coefficiente di film $h_g$, che può essere ricavato mediante la formula empirica di Bartz:

\begin{empheq}{equation*}\tag{*}
h_g = \left[ \frac{0.026}{D_t^{0.2}} \left( \frac{\mu ^{0.2} C_p}{Pr^{0.6}} \right)_{ns} \left(\frac{p_c g}{c*}\right)^{0.8} \left(\frac{D_t}{R}\right)^{0.1} \right] \left(\frac{A_t}{A} \right)^{0.9} \sigma
\qquad \cite{AIAA_book_1}\cite{AIAA_book_2}
\end{empheq}

Esso dipende, tra gli altri parametri, dal numero di Prandtl, dalla viscosità $\mu$, dal raggio di curvatura in gola dell'ugello e dal fattore correttivo $\sigma$, che tiene conto delle variazioni di proprietà fisiche attraverso lo strato limite. La viscosità del gas al valore di interesse di ristagno (ns) è definita in questo modo:
\begin{empheq}{equation*}\tag{*}
\mu = \left(46.6 \times 10^{-10} \right) Mm^{0.5} T^{0.6}
\end{empheq}
Dove $Mm$ è stato preso dalla simulazione RPA (\autoref{appendix:rpa}), ed è riferita al valore alla fine della camera di combustione.
Il raggio di curvatura in gola dell'ugello è stato ricavato tramite approssimazione di Rao. Il valore nelle parentesi quadre è costante, una volta fissata la geometria della camera come lo sono parametri di combustione ($c^*$, $p_c$). Il valore nelle parentesi con pedice 'ns' si riferisce ai valori di $\mu$, $c_p$ e $Pr$ all'inizio del convergente. I fattori moltiplicativi al di fuori delle parentesi quadre dipendono dalla posizione in cui si vuole calcolare tale coefficiente. Supponendo conosciuto il punto in cui si vuole calcolare $h_g$, l'unica incognita è quindi $\sigma$. Questo fattore può essere determinato in termini di temperatura di combustione, temperatura locale a parete e numero di mach locale mediante un ulteriore relazione di Bartz. In alternativa è possibile determinarlo per interpolazione, in funzione del rapporto $T_{wg} / T_c$ e del valore di $\gamma$ dai grafici forniti dalla fonte \cite{AIAA_book_1}.


Per il calcolo dello scambio termico nel caso di presenza di deposito solido sulle pareti della camera, l’equazione (*) viene corretta dalla seguente equazione, che vede una sostituzione del coefficiente di film con il coefficiente di conduttanza termica complessiva lato gas $h_{gc}$:

\begin{empheq}{equation*}
h_{gc} = \frac{1}{\frac{1}{h_g} + R_d}
\end{empheq}


Questo coefficiente considera sia $h_g$ sia il coefficiente di resistenza causata dal deposito solido $R_d$ , il cui valore è dipendente dal rapporto di espansione e dalle condizioni di pressione e rapporto di miscela (grafico specifico in \autoref{sec:raffreddamento}).
La trattazione del film cooling del tratto di ugello 10:16 è lasciata in \autoref{appendix:cooling_temp_definitions}.

\subsubsection{Regenerative cooling}
\label{subsubsec:regenerative cooling}

Il motore preso in esame sfrutta lo scambio termico rigenerativo come tecnica di raffreddamento delle pareti della camera di spinta, in particolare dalla gola e per la lunghezza dell'ugello fino al piano caratterizzato da rapporto di espansione 1:10. 

\rfig{temp_dist}{Variazione di temperatura attraverso la parete}{temp_dist}{0.3}

Il regenerative cooling utilizza una quota parte di combustibile, circa il 70\%, come refrigerante. Esso viene indirizzato in una serie di tubi opportunamente sagomati saldobrasati insieme che costituiscono la parete stessa dell'ugello di efflusso. Lo scambio di calore avviene quindi tra due flussi in movimento separati da una parete.Questa tecnica vanta di alcuni importanti vantaggi, tra i quali il fatto che non comporti nessuna perdita di prestazioni, infatti l'energia termica assorbita del refrigerante viene restituita all'iniettore, e abbia una struttura relativamente leggera. Tuttavia si possono riscontrare alcuni svantaggi, come alte perdite di pressione per elevati livelli di flusso di calore.
Il calore che attraversa la parete viene assorbito dal liquido refrigerante, il quale subisce un aumento di temperatura lungo il percorso. A causa degli effetti viscosi nasce uno strato limite sul lato coolant.
Il calore proveniente dal flusso di gas caldi è descritto dal parametro $h_{gc}$ analizzato precedentemente. Per mantenere la temperatura della parete entro valori contenuti, è necessario che la conduttività termica complessiva lato gas $h_{gc}$ sia minimizzata, mentre il coefficiente di scambio termico del refrigerante sia molto alto, così come il rapporto $t/k$. Dal momento che la differenza di temperatura è inversamente proporzionale al coefficiente di scambio termico del flusso di calore, la diminuzione della temperatura sarà più rapida tra gas caldo e parete interna della camera:

\begin{empheq}{equation*}
q = h_{gc} \left(T_{aw} - T_{wg}\right) = \frac{k}{t} \left(T_{wg} - T_{wc}\right) = h_c \left(T_{wc} - T_{co} \right)
\end{empheq}

Avendo che il flusso di calore attraverso i tre strati (strato limite gassoso, parete solida, strato limite liquido) è lo stesso, è possibile calcolare il parametro H (serie delle conduttività dei tre strati):
\vspace{3pt}

\begin{empheq}{gather*}
q = H\left(T_{aw} - T_{co}\right)
\qquad
H = \frac{1}{\frac{1}{h_{gc}} + \frac{1}{h_c} + \frac{t}{k}}
\end{empheq}

Il coefficiente $h_c$ descrive il processo di scambio termico attraverso lo strato limite del fluido refrigerante. Per determinarne il valore è necessario approfondire il suo legame con pressione e temperatura critica (vedi \autoref{appendix:cooling_temp_definitions}).
\subsubsection{Dimensionamento tubi camera di spinta}
L'obiettivo del regenerative cooling è quello di mantenere la temperatura della parete al di sotto della temperatura critica alla quale possono realizzarsi fusioni localizzate o un decremento delle prestazioni del materiale. La temperatura limite nel caso della parete della camera di spinta dell'F-1, realizzata in Inconel X750, è tra 550 K e 670 K.
Il dimensionamento del sistema di regenerative cooling è finalizzato a stabilire il numero di tubi che compongono la parete dell'ugello d'efflusso e le dimensioni dei singoli tubi, in particolare il diametro interno e lo spessore. Esso viene effettuato nella condizione più critica, ossia vengono dimensionati i tubi di ritorno nella sezione di gola, perché la gola rappresenta il punto caratterizzato dal maggior valore di flusso termico e attraverso la sezione finale dei tubi di ritorno scorre il refrigerante alla sua temperatura massima raggiunta dopo aver percorso tutto il sistema di raffreddamento. Il numero di tubi rimane costante fino al piano caratterizzato dal rapporto di espansione 1:3, per poi raddoppiare fino al piano con rapporto di espansione 1:10 (sdoppiamento spiegato in \autoref{subsec:descrizione camera spinta}). Avendo come variabili sia il numero di tubi che il loro spessore, è necessario fissare uno dei due dati. \'E stato scelto il numero reale di tubi che compongono l'ugello nel primo tratto (178).
I calcoli preliminari al dimensionamento permettono di determinare, tramite una trattazione analoga a quella illustrata per lo scambio convettivo, il valore di flusso di calore specifico q, funzione della conduttività termica, della temperatura adiabatica a parete e della temperatura della parete lato gas. 


\'E possibile passare al dimensionamento vero e proprio. Sono noti i valori relativi alla lega X750 (conducibilità termica $k_{lega}$, modulo di elasticità E, coefficiente di espansione termica a, coefficiente di Poisson $\nu$) e vengono assunti i valori di bulk temperature del combustibile in gola, la sua conducibilità termica $k_{fuel}$, la sua densità e una costante $C_1$ propria dell’RP-1 utile per il calcolo del numero di Nusselt. \cite{AIAA_book_1}\cite{AIAA_book_2}
Per determinare lo spessore t dei tubi è possibile implementare un ciclo in MATLAB che permetta di calcolare il numero dei tubi al variare dello spessore, per poi interrompere il ciclo quando il numero eguaglia il numero di tubi imposto. I calcoli svolti si basano su considerazioni fisiche e su formule empiriche 
\autoref{subsec:dimensionamento_tubi}.
\begin{table}[H]
\centering
\begin{tabular}{|c|c|c|c|c|c|c|c|c|}
\hline
$\bm{R_d \, [in^2sF/btu]}$ & $\bm{T_{wg} \, [K]}$ & $\bm{f_{aw} \, [-]}$  & $\bm{\sigma \, [-]}$  & $\bm{k_{fuel} \, [btu/insF]}$ & $\bm{T_{co} \, [K]}$ & $\bm{c_1 \, [-]}$ & $\bm{\dot{m} \, [kg/s]}$ & $\bm{N_{tubi} \, [-]}$ \\ 
\hline
$1125$ & $660$ & $0.93$  & $1.42$ & $1.78 \, \cdot \, 10^{-6}$ & $333.3$ & $0.0214$ & $556.9$ & $178$ \\
\hline
\end{tabular}
\caption{Dati usati per il dimensionamento dei tubi \cite{
AIAA_book_2}}
\label{table:tabella_regenerative_in}
\begin{tabular}{|c|c|c|c|c|}
\hline
$\bm{v_{co} \, [m/s]}$ & $\bm{h_{gc} \, [btu/in^2 sF]}$ & $\bm{\dot{q} \, [W/m^2]}$ & $\bm{d_{int} \, [cm]}$ & $\bm{t \, [mm]}$ \\
\hline
$44.2$ & $8.06\, \cdot \, 10^{-4} $ & $5.93\, \cdot \, 10^6 $ & $  1.49 $ & $0.81$ \\
\hline
\end{tabular}
\caption{Risultati dimensionamento tubi}
\label{table:tabella_regenerative_out}
\end{table}

\subsubsection{Grafici andamento flusso di calore e temperatura}

In questo paragrafo verranno presentati i grafici degli andamenti del flusso di calore e dell'andamento delle temperature $T_{aw}$, $T_{wg}$ e $T_c$. In particolare viene considerato il tratto in cui è effettuato il raffreddamento rigenerativo. La temperatura del coolant viene considerata solo nel condotto di ritorno (risale l'ugello) per semplicità. Il metodo iterativo per il calcolo di tali grandezze è stato tratto dal manuale sulla trattazione dei raffreddamenti delle camere di spinta di LRE di RPA (\cite{rpa_thermal_manual}) ed è stato implementato in \textit{Matlab}. \autoref{appendix:codici}

\twofigII{heat_flux}{Flusso di calore in funzione della posizione assiale}{heat_flux}{temperature_cooling}{Temperature di interesse in funzione della posizione assiale}{temperature_cooling}{0.7}

Da \autoref{fig:heat_flux} si vede che il flusso di calore è massimo in gola. Da \autoref{fig:temperature_cooling} si nota come $T_{wg}$ sia minore della temperatura di fusione del materiale. Sempre nella stessa figura si può notare che lo sdoppiamento dei tubi genera un salto nel grafico delle temperature.