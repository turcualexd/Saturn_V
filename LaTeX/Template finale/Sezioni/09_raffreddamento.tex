\section{Sistemi di raffreddamento}
\label{sec:raffreddamento}

\subsection{Cooling della camera di spinta}
\label{subsec:cooling camera}

I motori a propellente liquido sfruttano varie tecnologie per il raffreddamento delle pareti della camera di spinta. Nel caso analizzato, il motore F-1 sfrutta due tipi di raffreddamento: il film cooling, che protegge le pareti dell'estensione dell'ugello attraverso un sistema di iniezione, e il raffreddamento rigenerativo, che utilizza il combustibile come fluido refrigerante passante attraverso una serie di tubi che costituiscono la parete stessa dell'ugello.

\subsubsection{Scambio termico convettivo e film cooling}

\rfig{boundary_layer}{Strato limite sulla parete dell'ugello}{boundary_layer}{0.5}

Per poter analizzare la protezione termica delle pareti della camera di spinta, è in primo luogo necessario stimare il valore di scambio termico convettivo dai gas combusti alle pareti stesse. La trattazione dello scambio termico convettivo nel caso preso in analisi viene affrontata tenendo conto dalle alte velocità dei gas combusti: ciò porta alla formazione di uno strato limite, che si assottiglia lungo il convergente in concomitanza con l'accelerazione del fluido subsonico, raggiungendo il minimo in gola, per poi ispessirsi nel divergente. Lo scambio termico è quindi un problema riguardante lo strato limite e il suo spessore, la sua temperatura e la velocità del fluido. Poichè si raggiunge il minimo spessore dello strato limite in gola, ci si aspetta di avere il massimo scambio convettivo nel punto dell'ugello in cui il rapporto $A_t/A$ è minimo. Questa osservazione è di particolare rilevanza, poiché, come si vedrà più avanti, per determinare la portata massica necessaria per il film cooling, si utilizzerà l'area della sezione minima dell'estensione dell'ugello, che corrisponde all'area della sezione con rapporto di espansione 10:1.

Risulta complicato determinare il valore preciso del calore scambiato in modo convettivo tra i gas combusti e le pareti, in quanto lo strato limite è fortemente influenzato da vari fattori, quali la curvatura delle pareti, il gradiente di pressione in direzione assiale, il gradiente di temperatura associato all'alta intensità del flusso di calore; è tuttavia possibile utilizzare un metodo semi-empirico per fare una stima accurata.

Lo scambio convettivo per unità di area lato gas,all’interfaccia tra fluido e superficie solida, è descritto dal coefficiente di film $h_g$:
\vspace{2pt}
\begin{empheq}{equation*}
q = h_g \left( T_{aw} - T_{wg} \right)
\end{empheq}
\vspace{3pt}
Per determinare la temperatura $T_{wg}$ è sufficiente moltiplicare la temperatura in camera di combustione per un fattore pari a 0.8, fattore che tiene conto della presenza di depositi solidi di carbonio sulle pareti (tale valore verrà giustificato in questa trattazione).
Per comprendere il significato della temperatura adiabatica $T_{aw}$ è necessaria una piccola digressione trattata in \autoref{appendix:cooling_temp_definitions}.

Nel caso preso in esame, il valore della temperatura $T_{aw}$ è determinabile scalando la temperatura in camera di combustione di un fattore detto "recovery factor" $f_{aw}$ , definito come legame tra $T_{aw}$ e le temperature statica e totale del flusso libero e con valore compreso tra 0.9 e 0.98. In particolare il recovery factor rappresenta il rapporto tra l'aumento della temperatura causato dall'attrito e l'aumento causato dalla compressione adiabatica. Esso è determinabile sperimentalmente o può essere stimato tramite la sequente relazione
\begin{empheq}{equation*}
f_{aw} = \frac{1 + r \left(\frac{\gamma - 1}{2} M_x^2\right)}{1 + \left(\frac{\gamma - 1}{2} M_x^2\right)} 
\qquad
r = Pr^{0.33};
\end{empheq}
È necessario precisare che la temperatura in camera di combustione $T_c$ utilizzata è quella teorica moltiplicata per il fattore correttivo della velocità caratteristica elevato al quadrato. Quest'ultima velocità infatti dipende unicamente dalla variabile $\sqrt{T_c}$. Il suddetto fattore varia in un intervallo compreso tra 0.87 e 1.03, mentre il valore utilizzato nella trattazione, ossia 0.975, è il valore sperimentale adottato dalla fonte \cite{AIAA_book_1}

Per stabilire il valore di calore scambiato per unità di area rimane da calcolare solo il coefficiente di film $h_g$, che può essere ricavato mediante la formula empirica di Bartz:

\begin{empheq}{equation*}\tag{*}
h_g = \left[ \frac{0.026}{D_t^{0.2}} \left( \frac{\mu ^{0.2} C_p}{Pr^{0.6}} \right)_{ns} \left(\frac{p_c g}{c*}\right)^{0.8} \left(\frac{D_t}{R}\right)^{0.1} \right] \left(\frac{A_t}{A} \right)^{0.9} \sigma 
\end{empheq}

\vspace{5pt}
\cite{AIAA_book_1}
\cite{AIAA_book_2}

Esso dipende, tra gli altri parametri, dal numero di Prandtl, dalla viscosità $\mu$, dal raggio di curvatura in gola dell'ugello e dal fattore correttivo $\sigma$, che tiene conto delle variazioni di proprietà fisiche attraverso lo strato limite. La viscosità del gas al valore di interesse di ristagno (ns) è definita in questo modo:
\begin{empheq}{equation*}\tag{*}
\mu = \left(46.6 \times 10^{-10} \right) Mm^{0.5} T^{0.6}
\end{empheq}
Dove $Mm$ è stato presa dalla simulazione RPA (appendice da fare), ed è riferita al valore alla fine della camera di combustione. 
Il raggio di curvatura in gola dell'ugello è stato ricavato tramite approssimazione di Rao. Il valore nelle parentesi quadre è costante, una volta fissata la geometria della camera e i parametri di combustione ($c*$, $p_c$). Il valore nelle parentesi con pedice 'ns' si riferisce ai valori di $\mu$, $c_p$ e $Pr$ all'inizio del convergente. I fattori moltiplicativi al di fuori delle parentesi quadre dipendono dalla posizione in cui si vuole calcolare tale coefficiente. Supponendo conosciuto il punto in cui si vuole calcolare $h_g$, l'unica incognita è  quindi $\sigma$. Questo fattore può essere determinato in termini di temperatura di combustione, temperatura locale a parete e numero di mach locale mediante un ulteriore relazione di Bartz; in alternativa è possibile determinarlo per interpolazione, in funzione del rapporto $T_{wg} / T_c$ e del valore di $\gamma$ dai grafici forniti dalla fonte bibliografica \cite{AIAA_book_1}

\rfig{sigma}{Valore di $\sigma$}{sigma}{0.4}

Assumendo il rapporto $T_{wg}/T_c$ pari a 0.8, ricavato sperimentalmente e adottato nella trattazione dello stesso tipo di dimensionamento nel riferimento \cite{AIAA_book_2}, si tiene conto della presenza del deposito di carbonio sulle pareti. Noto il valore di $\gamma$, pari a 1.1754, e ricordando che il fine ultimo del calcolo è progettare il film cooling dell'ugello aggiuntivo (intervallo di rapporto di espansione $\epsilon=$ 10/16), è possibile determinare dal grafico che il valore del fattore correttivo si attesta intorno a 0.7 in tutto l'intervallo in esame.
Il coefficiente di film dipende infine anche dal rapporto $A_f/A$, dove A è l'area della sezione locale. Il valore di questo rapporto è stato fatto variare per via numerica tra 10 e 16, calcolando poi per ciascun valore il corrispondente coefficiente di film e, in seguito, la corrispondente portata minima in massa per effettuare un adeguato film cooling.
Il valore di $h_g$ così ottenuto tiene unicamente in conto del calore scambiato tra fluido e parete, senza considerare la presenza di eventuali prodotti di combustione allo stato solido. I prodotti di combustione della coppia LOX – RP-1 provenienti dalla combustione del Gas Generator, introdotti sulla parete dell'ugello aggiuntivo dal rapporto di espansione 10 fino all'efflusso, contengono circa il 37\% di particolato solido $C_{graf}$. Tale dato è stato  ottenuto tramite analisi CEA, presentata in \autoref{appendix:prodotti_gas_generator}). Queste particelle tendono a depositarsi sulle pareti della camera di combustione, formando un efficace strato isolante: la valutazione quantitativa dell’efficacia dell’isolamento di questo strato, necessaria per il corretto calcolo dello scambio di calore, può essere effettuata solo sperimentalmente. Lo strato isolante è formato a sua volta da uno strato superficiale di fuliggine, che ne sovrasta uno piu tenace: quest’ultimo aumenta la resistenza termica lato gas, tale che la temperatura del deposito di carbonio all’interfaccia lato gas si avvicini alla temperatura del gas all’aumentare dello spessore del layer di carbonio (questo giustifica qualitativamente il valore di 0.8 considerato per il rapporto di $T_{wg}/T_c$).
Per il calcolo dello scambio termico nel caso di presenza di deposito solido sulle pareti della camera, l’equazione (*) viene corretta dalla seguente equazione, che vede una sostituzione del coefficiente di film con il coefficiente di conduttanza termica complessiva lato gas $h_{gc}$

\begin{empheq}{equation*}
h_{gc} = \frac{1}{\frac{1}{h_g} + R_d}
\end{empheq}


Questo coefficiente considera sia $h_g$ sia il coefficiente di resistenza causata dal deposito solido $R_d$ , il cui valore è dipendente dal rapporto di espansione e dalle condizioni di pressione e rapporto di miscela (grafico specifico in \autoref{sec:raffreddamento}).
Dopo aver calcolato tutti i parametri necessari, è quindi possibile progettare il sistema di film cooling dell'estensione dell'ugello. Il film cooling delle pareti interne è ottenuto iniettando i gas di scarico della turbina, forniti alla cavità tra le pareti dal collettore di scarico della turbina, nel flusso di scarico della camera di spinta attraverso fessure formate da 23 file di scandole sovrapposte che formano la parete interna.(mettere una immagine per far capire).
Per lo sviluppo dei calcoli si consideri che il fluido di lavoro è gas con presenza di particolato, ed è quindi possibile utilizzare la relazione di Hatch e Papell, sostituendo al coefficiente $h_g$ il coefficiente $h_{gc}$ appena calcolato.

\vspace{5pt}
\begin{empheq}{equation*}
\frac{T_{aw} - T_{wg}}{T_{aw} - T_{co}} = \exp\left[  -\left(  \frac{h_{gc}}{G_c C_{pvc} \eta_c} \right) \right]
\end{empheq}
\vspace{5pt}

Dove $T_{co}$ è la temperatura iniziale del fluido refrigerante, ossia la temperatura all'uscita dello scambiatore; $C_{pvc}$ è il calore specifico medio a pressione costante del fluido refrigerante, che è stato numericamente ottenuto interpolando i valori dopo la combustione nel Gas Generator in frozen equilibrium; infine $\eta_{c}$ è l'efficienza del film cooling ed è un fattore che ha scopo correttivo, ossia tiene conto della quantità di refrigerante gassoso perso nel flusso di gas di combustione che quindi non produce effetti di raffreddamento. I valori dell'efficienza variano dal 25 al 65 in percentuale in funzione della geometria dell'iniezione del refrigerante e dalle condizioni di flusso. Il valore $G_c$ rappresenta la portata di gas che costituisce il film cooling necessaria a ottenere il raffreddamento voluto.
Dalla precedente equazione si evince che l'apporto termico dipende dal coefficiente di scambio $h_gc$ e dalla differenza tra temperatura adiabatica a parete e la temperatura del refrigerante; il calore assorbito è proporzionale alla capacità termica del film refrigerante dal valore di temperatura iniziale a quello finale. Esiste quindi un equilibrio tra apporto di calore e aumento di temperatura del refrigerante: raggiunto questo equilibrio si raggiunge la condizione adiabatica e la superficie della parete avrà localmente la medesima temperatura del film; infatti la temperatura della parete varierà assialmente dalla temperatura iniziale del refrigerante fino alla temperatura massima ammissibile.
L'obiettivo del calcolo è perciò quello di determinare la portata massica di fluido refrigerante per unità di area $G_c$, che poi verrà moltiplicato per l'area dell'estensione dell'ugello ad ottenere il valore di portata massica necessaria per il film cooling. Si noti che la portata dipende dal valore $h_{gc}$ , a sua volta dipendente dal rapporto $A_t/A$, che è stato fatto variare tra 1:10 e 1:16 : la portata massica che sarà sufficiente a raggiungere un efficiente film cooling in ogni sezione dell'ugello sarà la portata massima tra le portate calcolate, ossia quella ottenuta per rapporto $A_t$ su A maggiore e perciò A minore, quindi l'area della sezione 1:10. Il valore di $G_c$ ottenuto è minore della portata elaborata dal Gas Generator, e questo è un risultato prevedibile in quanto il valore di portata passante per il Gas Generator è dettato dai requisiti di potenza della turbina e non dalle esigenze del film cooling. È stato perciò dimostrato che la portata massica elaborata è sufficiente a raggiungere l'obiettivo desiderato di raffreddamento delle pareti.

\subsubsection{Regenerative cooling}
\label{subsubsec:regenerative cooling}

Il motore preso in esame sfrutta lo scambio termico rigenerativo come tecnica di raffreddamento delle pareti della camera di spinta, in particolare dalla gola e per la lunghezza dell'ugello fino al piano caratterizzato da rapporto di espansione 10:1. Il regenerative cooling utilizza una quota parte di combustibile stivato, circa il 70\%, come refrigerante: esso viene indirizzato in una serie di tubi opportunamente sagomati saldobrasati insieme che costituiscono la parete stessa dell'ugello di efflusso. Lo scambio di calore avviene quindi tra due flussi in movimento separati da una parete.
Questa tecnica vanta di alcuni importanti vantaggi, tra i quali il fatto che non comporti nessuna perdita di prestazioni, infatti l'energia termica assorbita del refrigerante viene restituita all'iniettore, e abbia una struttura relativamente leggera. Tuttavia si possono riscontrare alcuni svantaggi, come alte perdite di pressione per elevati livelli di flusso di calore.

\rfig{temp_dist}{Variazione di temperatura attraverso la parete}{temp_dist}{0.4}

La figura descrive la variazione di temperatura durante lo scambio di calore per regenerative cooling: a sinistra scorrono i gas combusti a contatto con il boundary layer e la cui temperatura è $T_{aw}$ che diminuisce sensibilmente all'interno del boundary layer fino a raggiungere la temperatura della parete lato gas $T_{wg}$. All'interno dello spessore della parete la temperatura continua a diminuire raggiungendo la temperatura $T_{wc}$, ossia la temperatura della parete a contatto con il refrigerante; quest'ultimo quindi sarà caratterizzato dalla temperatura $T_{c0}$ (bulk temperature del refrigerante) (scrivere definizione).

Proprio a causa dello scambio di calore tra gas e refrigerante, la temperatura $T_{c0}$ aumenterà dal punto di ingresso fino al momento in cui l'RP-1 lascerà il condotto di raffreddamento: essa è quindi una funzione del calore assorbito e della portata. A livello strutturale è necessario svolgere il dimensionamento nel punto più critico, ossia la sezione all'altezza della gola del tubo di ritorno, poichè corrisponde all'ultima sezione attraversata dal refrigerante prima di essere immesso nella camera di spinta.

L'obiettivo ultimo del regenerative cooling è quello di mantenere la temperatura della parete al di sotto della temperatura critica alla quale possono realizzarsi fusioni localizzate o un decremento delle prestazioni del materiale. La temperatura limite nel caso della parete della camera di spinta dell'F-1, realizzata in Inconel X750, è tra 550 K e 670 K.
Identificate le temperature caratteristiche del processo di raffreddamento è quindi possibile calcolare il flusso di calore come:

\vspace{5pt}
\begin{empheq}{equation*}
q = h_{gc} \left(T_{aw} - T_{wg}\right) = \frac{k}{t} \left(T_{wg} - T_{wc}\right) 
\end{empheq}
\vspace{3pt}
Avendo che il flusso di calore attraverso i tre strati (strato limite gassoso, parete solida, strato limite liquido) è lo stesso, posso calcolare il parametro H (serie delle conduttività dei tre strati)
\vspace{3pt}
\begin{empheq}{equation*}
q = H\left(T_{aw} - T_{co}\right) 
\end{empheq}
\begin{empheq}{equation*}
H = \frac{1}{\frac{1}{h_{gc}} + \frac{1}{h_c} + \frac{t}{k}}
\end{empheq}
\vspace{5pt}
Dove $h_{gc}$ è la conduttività termica complessiva lato gas, $h_{c}$ è il coefficiente di scambio termico lato refrigerante, mentre t è lo spessore della parete e k la conduttività termica della parete della camera. Osservando la precedente equazione è possibile introdurre un ulteriore requisito che il regenerative cooling deve soddisfare: per mantenere la temperatura della parete entro valori contenuti, è necessario che la conduttività termica complessiva lato gas $h_{gc}$ sia minimizzata, mentre il coefficiente di scambio termico del refrigerante sia molto alto, cosi come il rapporto $t/k$. Dal momento che la differenza di temperatura è inversamente proporzionale al coefficiente di scambio termico del flusso di calore, la diminuzione della temperatura sarà più rapida tra gas caldo e parete interna della camera.

Se per determinare il valore di $h_{gc}$ è sufficiente ripercorrere la trattazione riguardante lo scambio termico convettivo, per comprendere il significato del coefficiente $h_c$ e determinarne il valore numerico è necessario approfondire il suo legame con pressione e temperatura critica del refrigerante (vedi \autoref{sec:raffreddamento})

Per i motivi illustrati in appendice si predilige avere una pressione che sia tra il 30 e il 70 \% della pressione critica.
Il dimensionamento del sistema di regenerative cooling è finalizzato a stabilire il numero di tubi che compongono la parete dell'ugello d'efflusso e le dimensioni dei singoli tubi, in particolare il diametro interno e lo spessore. Prima di procedere alla trattazione matematica è necessario chiarire alcune assunzioni considerate durante lo svolgimento dei calcoli. Il dimensionamento viene effettuato nella condizione più critica, ossia vengono dimensionati i tubi di ritorno nella sezione di gola, perché la gola rappresenta il punto caratterizzato dal maggior valore di flusso termico e attraverso la sezione finale dei tubi di ritorno scorre il refrigerante alla sua temperatura massima raggiunta dopo aver percorso tutto il sistema di raffreddamento. La forma dell'ugello d'efflusso fa fede alla modellazione illustrata precedentemente e viene perciò considerata nota: dalla modellazione e dalla simulazione RPA verranno ricavati il diametro di gola e i raggi di curvatura utili a determinare il raggio di curvatura medio R; il numero di tubi rimane costante fino al piano caratterizzato dal rapporto di espansione 3:1, per poi raddoppiare fino al piano con rapporto di espansione 10:1 (sdoppiamento spiegato in \autoref{subsec:descrizione camera spinta}). Infine, avendo come variabili sia il numero di tubi sia il loro spessore, è necessario ipotizzare o fissare uno dei due dati: è stato quindi fissato il numero reale di tubi che compongono l'ugello nel primo tratto, ossia 178, mantenendo come incognita lo spessore.
I calcoli preliminari al dimensionamento permettono di determinare, tramite una trattazione analoga a quella illustrata per lo scambio convettivo, il valore di flusso di calore specifico q, funzione della conduttività termica, della temperatura adiabatica a parete e della temperatura della parete lato gas. 
La temperatura a parete lato gas $T_{wg}$ è determinata sperimentalmente, mentre la temperatura adiabatica a parete $T_{aw}$ è ottenuta moltiplicando la temperatura in camera di combustione $T_c$ per il fattore di recupero dello strato limite turbolento in gola (valore intermedio tra 0.9 e 0.98). Noto quindi il rapporto $T_{wg}/T_c$ e $\gamma$ dei gas combusti è possibile determinare il valore del fattore di correzione in gola $\left( \sigma \right)$ dai grafici riportati nella sezione precedente. Infine è possibile calcolare il coefficiente di scambio termico lato gas tramite la formula (*)
e quindi il valore di $h_{gc}$

\rfig{nozzle_definitions}{Definizioni grandezze geometriche ugello}{nozzle_definitions}{0.5}

Dalle precedenti formule è possibile definire $R_d$ la resistenza termica causata dal deposito solido in gola, R il raggio di curvatura dell'ugello calcolato come media dei due raggi di curvatura $R_1$ ed $R_n$, $D_t$ il diametro di gola calcolato come due volte $R_t$.
Terminati i calcoli preliminari è possibile passare al dimensionamento vero e proprio del sistema di refrigerazione. Sono noti i valori relativi alla lega X750 (conducibilità termica $k_{lega}$, modulo di elasticità E, coefficiente di espansione termica a, coefficiente di Poisson $\nu$) e vengono assunti i valori di bulk temperature del combustibile in gola, la sua conducibilità termica $k_{fuel}$, la sua densità (di stivaggio) e una costante $C_1$ propria dell’RP1 utile per il calcolo del numero di Nusselt. \cite{AIAA_book_1}\cite{AIAA_book_2}
A questo punto per determinare lo spessore t dei tubi è possibile implementare un ciclo in \textit{Matlab} che permetta di calcolare il numero dei tubi al variare dello spessore, per poi interrompere il ciclo quando il numero eguaglia il numero di tubi imposto: in questo modo si ottiene il valore dello spessore necessario. I calcoli svolti si basano su considerazioni fisiche e su formule empiriche.
Il ciclo inizia con il calcolo della temperatura della parete lato combustibile.
\vspace{3pt}
\begin{empheq}{equation*}
T_{wc} = T_{wg} - \frac{qt}{k}
\end{empheq}
\vspace{3pt}
Necessario per determinare il valore del coefficiente  di scambio termico del combustibile $h_c$
\vspace{3pt}
\begin{empheq}{equation*}
h_{c} = \frac{q}{T_{wc} - T_{co}}
\end{empheq}
\vspace{3pt}
Per regioni caratterizzate da temperatura subcritica e dall'assenza di nucleate boiling, la relazione tra temperatura a parete e flusso di calore, che dipende per l'appunto dal coefficiente di scambio termico $h_c$, può essere descritta tramite l'equazione di Sieder-Tate per il trasferimento turbolento di calore a liquidi che fluiscono nei canali: 
\vspace{3pt}
\begin{empheq}{equation*}
Nu = C_1 Re^{0.8} Pr^{0.4} \left(\frac{\mu}{\mu_w}\right)^{0.14}
\end{empheq}
\vspace{3pt}
Dove $\mu$ è la viscosità del combustibile alla temperatura $T_{co}$, mentre $\mu _w$ è la viscosità del combustibile alla temperatura della parete in gola. Questa relazione può essere riscritta esplicitando i singoli termini:
\vspace{3pt}
\begin{empheq}{equation*}\tag{E1}
Nu = \frac{h_c d}{k}=C_1 \left(\frac{\rho V_{co} d}{\mu}\right)^{0.8} \left(\frac{\mu C_p}{k}\right)^{0.4} \left(\frac{\mu}{\mu_w}\right)^{0.14}
\end{empheq} 
\vspace{3pt}
Dove le incognite sono il diametro dei tubi d e la velocità media del combustibile $V_{co}$. Quest'ultima può essere calcolata in funzione del diametro dei tubi e del loro numero
\vspace{3pt}
\begin{empheq}{equation*}
V_{co} = \frac{\frac{\dot{W}_f}{\rho}}{\frac{N}{2} \frac{\left(\pi d^2\right)}{4}}
\end{empheq}
\vspace{3pt}
Con $\dot{W}_f$ la portata massica di combustibile, corrispondente al 70 \% della portata totale del combustibile.
Il numero di tubi è ottenuto, qualitativamente, dividendo il perimetro dell'ugello della sezione di gola e il diametro del tubo considerato. Raffinando la formula con fattori correttivi si trova la seguente formula empirica
\vspace{3pt}
\begin{empheq}{equation*}\tag{E2}
N = \frac{\pi \left[D_t + 0.8\left(d + 0.04 \right)\right]}{d + 0.04}
\end{empheq}
\vspace{3pt}

Il fattore 0.8 ha il ruolo di fattore di correzione: il centro dei tubi è collocato su una circonferenza, piuttosto che su una retta.
Sostituendo quindi $V_{co}$ all'interno dell'equazione (E1) esplicitandone N ed eguagliando l'espressione trovata all'equazione (E2) è possibile determinare il valore del diametro dei tubi. Sostituendo infine il valore trovato all'interno dell'equazione (E2) il si ottiene il numero dei tubi. Analizzando in un 'ciclo for' i passaggi appena visti, è possibile determinare il valore di t tale per cui si ha il numero di tubi N desiderato.
