\begin{abstract}
\addcontentsline{toc}{section}{Sommario}
\vspace*{5mm}

La presente relazione di prova finale intende dare una descrizione dell'endoreattore F-1 prodotto da Rocketdyne. Cinque di questi motori vennero installati sul primo stadio S-IC del vettore Saturn V che portò il primo uomo sulla luna. L'obiettivo di questo stadio era quello di portare il razzo ad una quota di 61 km, fornendo un $ \Delta v \simeq $ 2300 m/s.\\
Di seguito verranno analizzati i principali sistemi per un singolo motore, partendo dal sistema di alimentazione, passando per il sistema di generazione della potenza ed arrivando infine al sistema di espansione gasdinamico e al suo raffreddamento. Si provvederà inoltre a dare una descrizione quali/quantitativa delle scelte progettuali applicate ai tempi. Infine, verrà studiata una alternativa ai propellenti utilizzati, rimarcando le conseguenze sull'intero sistema propulsivo che tale variazione implica.

\end{abstract}