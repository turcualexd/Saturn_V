\section{RPA: definizione problema e funzionamento}
\label{appendix:rpa}

RPA è un software che permette di prevedere le prestazioni per motori a razzo. Il metodo di calcolo si basa su un approccio di minimizzazione dell'energia libera di Gibbs al fine di ottenere la composizione dei prodotti di combustione, l'analisi del flusso dell'ugello con shifting o frozen equilibrium e il calcolo delle prestazioni del motore, per camera di combustione finita o infinita.

RPA utilizza una libreria di specie chimiche espandibile basata sul database termodinamico NASA Glenn, che include dati per numerosi combustibili e ossidanti. È inoltre possibile definire nuovi componenti del propellente o importare componenti da database di specie PROPEP o CEA2.
Fornendo parametri propri del motore in analisi, come la pressione della camera di combustione, i componenti del propellente utilizzati e i parametri dell'ugello, il programma permette di ottenere la composizione di equilibrio chimico dei prodotti di combustione, determina le sue proprietà termodinamiche e predice le prestazioni teoriche del razzo. 

Il software è organizzato in queste macroaree: 

• \textbf{Initial Data}

Questa sezione contiene 3 sottosezioni:
\begin{itemize}
\item \textit{Engine Definition}: in questa sottosezione deve essere definito il nome del problema con una breve descrizione. Si introduce il valore della pressione della camera di combustione. Si definisce la grandezza del sistema fissando uno tra i seguenti parametri: la spinta e la pressione cui è riferita, la portata massica oppure il diametro di gola. Si è scelto di fissare come parametro di ingresso la spinta poichè rappresenta un requisito da soddisfare. Infine, si definisce il sistema di alimentazione che nel nostro caso è un sistema a turbopompa.
\item \textit{Propellant Specification} Per caratterizzare il propellente è necessario definire il tipo di sistema, in questo caso bipropellente, il rapporto di miscela O/F, l'ossidante e il combustibile con le relative temperature di stivaggio. 
\item \textit{Nozzle Flow Model}: in questa sottosezione si definisce il modello di camera di combustione (infinita o finita) definendo il contraction ratio o la portata massica per unità di area all'inizio del convergente. La  condizione di uscita è descritta dal rapporto di espansione, come nel nostro caso, oppure attraverso la pressione all'efflusso. Infine, si definisce il modello di reazione chimica utilizzato. In particolare, viene assunto shifting equilibrium fino ad un punto prefissato. Dopo quel punto, definito in base a $A_{fr}/A_t$ o $p_t/p_{fr}$si applica il modello frozen equilibrium.
\end{itemize}

• \textbf{Performance Analysis}

Questa sezione contiene alcuni output. Illustra le prestazioni della camera, in particolare le proprietà termodinamiche (ovvero le percentuali in massa dei prodotti di combustione in vari punti del motore) e le prestazioni stimate dell'endoreattore. Nel nostro caso abbiamo riportato nella (tabella da mettere) le percentuali in massa, nella tabella \autoref{table:valori RPA sezione performance analysis} alcuni valori termodinamici in alcune sezioni del motore. E nella tabella (da mettere) le performance del motore.

• \textbf{Engine Design}

Questa macro-area è divisa in tre sezioni: \textit{Chamber geometry}, \textit{Thermal analysis} e \textit{Propellant feed system}.
All'interno della prima sezione è possibile fissare il design della camera: per l'analisi sono stati fissati i valori di angolo di contrazione b, i rapporti tra i vari raggi di curvatura e l'approssimazione parabolica dell'ugello a campana. Inoltre è possibile visualizzare le grandezze di spinta, impulso e portata e i valori che riguardano la geometria della camera.
All'interno della sezione \textit{Thermal analysis} è possibile studiare lo scambio termico convettivo, l'irraggiamento e il cooling. Questa sezione non è stata utilizzata, prediligendo uno studio del sistema di raffreddamento numerico (\autoref{sec:raffreddamento}).
Dall'ultima sezione, \textit{Propellant feed system} è possibile ottenere il valore dei parametri di funzionamento dei vari sistemi di alimentazione, quali portata, temperatura, pressione in ingresso e in uscita. Nel caso in esame si è scelto di studiare il sistema di alimentazione di ossidante, combustibile e del gas generator. (TABELLA DI VALORI). Per ottenere gli output sopra citati è necessario inserire varie informazioni sul gas generator e sulle turbopompe precisate nella tabella (da mettere). Per quanto riguarda il gas generator, è stato specificato che il ciclo in esame è Fuel Rich, e vanno indicati i valori di temperatura e di pressione di combustione, rispettivamente 1062.59 K e 67.5686 bar, oltre al coefficiente di perdita di pressione, pari a 0.966. La turbopompa è invece caratterizzata dal rapporto di pressione della turbina e dall'efficienza, rispettivamente 16.293 e 0.605. 


\begin{table}[H]
\centering
\begin{tabular}{|c|c|c|c|c|c|}
\hline
& \textbf{Injector} & \textbf{Nozzle inlet} & \textbf{Nozzle throat} & \textbf{Nozzle exit} & \textbf{Unit} \\
\hline
\textbf{p} & $7.7566$ & $5.9496$ & $3.9690$ & $0.0427$ & \text{MPa} \\
\hline
\textbf{T} & $3569.9063$ & $3498.7671$ & $3344.5972$ & $1473.2666$ & \text{K} \\
\hline
$\bm{c_p}$ & $4.7323$ & $4.7034$ & $0.3706$ & $0.3678$ & \text{kJ/kgK} \\
\hline
$\bm{\gamma}$ & $1.1777$ & $1.1756$ & $1.1716$ & $1.2521$ & \text{-} \\
\hline
\textbf{MM} & $22.2095$ & $22.2541$ & $22.4347$ & $22.6040$ & \text{lb/mol} \\
\hline
\textbf{M} & $0$ & $0.5137$ & $1$ & $3.6856$ & \text{-} \\
\hline
\end{tabular}
\caption{Proprietà termodinamiche in output da RPA - sezione performance analysis }
\label{table:valori RPA sezione performance analysis}
\end{table}

\vspace{10pt}

\begin{table}[H]

\centering
\begin{tabular}{|c|c|c|c|c|}
\hline
& $\bm{\dot{m} \, [kg/s]}$ & $\bm{p_{in} \, [MPa]}$ & $\bm{p_{out} \, [MPa]}$ \\
\hline
\textbf{Inlet} & $1769.26$ & $0.448$ & $0.375$ \\
\hline
\textbf{Pump} & $1769.26$ & $0.375$ & $8.627$ \\
\hline
\textbf{Injector} & $1756.218$ & $8.257$ & $7.757$ \\
\hline
\end{tabular}
\caption{Sistema di alimentazione del propellente (output RPA) - Sezione Engine Design }
\label{table:Sistema di alimentazione del propellente RPA - sezione engine design}
\end{table}


\pagebreak
