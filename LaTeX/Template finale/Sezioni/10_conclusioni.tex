\section{Conclusioni}
\label{sec:conclusioni}
Nella trattazione effettuata i valori numerici ottenuti dai dimensionamenti sono in linea con i dati forniti dai manuali dello stadio. I vari risultati presentati sono stati discussi nei capitoli di interesse e relative appendici.\\
L'intera trattazione è stata basata prevalentemente sulla consultazione di manuali dell'epoca forniti direttamente dalla NASA. Tali manuali contengono principalmente criteri di design che vennero raccolti direttamente dai report degli ingegneri che lavoravano al progetto. I manuali sono divisi in vari volumi in cui si presenta il componente di interesse e esposti i problemi che sono stati riscontrati durnte la progettazione.\\
Il lavoro svolto si è concentrato sull'analisi approfondita dei singoli componenti, piuttosto che sulla sostituzione con altri elementi di fattura moderna, che avrebbero portato ad un ipotetico miglioramento delle prestazioni ma provocando anche un profondo cambiamento a livello strutturale, denaturalizzando la visione di insieme del progetto.
Il dimensionamento più completo dovrebbe considerare il funzionamento delle varie componenti anche in fase di avvio e spegnimento. In un contesto progettuale vero e proprio, alla sola analisi numerica, si accompagnano diverse prove su banco. In tempi più moderni rispetto al periodo di progettazione del suddetto stadio, si fa principalmente ricorso a software di simulazione più avanzata per quanto riguarda sia aspetti chimici e fluidodinamici dell'espansione gasdinamica e dei processi di  combustioni, sia per aspetti di ottimizzazione strutturale delle componenti. 