\section{Camera di spinta}
\label{appendix:camera_spinta}

\subsection{Piatto d'iniezione}

\cfig{inj_1}{Rappresentazione piatto d'iniezione \cite{engine_manual_supplement}}{inj_1}{0.5}
\cfig{inj_3}{Rappresentazione piatto d'iniezione \cite{injector}}{inj_3}{0.45}


\subsection{Modellazione ugello}

\cfig{angoli_bell}{Grafico di interpolazione degli angoli per la costruzione dell'ugello}{angoli_bell}{0.5}
\twofig{2DNozzle_1_10}{Costruzione ugello 2D 1:10}{2DNozzle_1_10}{3DNozzle_1_10}{Costruzione ugello 3D 1:10}{3DNozzle_1_10}
\twofig{2DNozzle_1_16}{Costruzione ugello 2D 1:16}{2DNozzle_1_16}{3DNozzle_1_16}{Costruzione ugello 3D 1:16}{3DNozzle_1_16}

\subsection{Confronto tra ugello 10:1 e 16:1}

I motori F-1 prodotti dalla Rocketdyne avevano in origine un ugello il cui rapporto di espansione era 10:1; essi infatti non furono inizialmente progettati nello specifico per lo stadio di lancio del Saturn V. Gli ingegneri decisero quindi a posteriori di aggiungere un'espansione dell'ugello iniziale allo scopo di migliorare vari parametri del lanciatore: i più rilevanti sono l'impulso specifico nel vuoto e la quota a cui è raggiungibile l'espansione ottima (adattata alla traiettoria che il lanciatore avrebbe percorso).

Tale espansione non poteva essere raffreddata dal già presente regenerative cooling: si optò dunque per una soluzione che prevedesse l'utilizzo dei gas di scarico della turbina, ricchi di carbonio e quindi con bassa conducibilità termica, per il raffreddamento attraverso film cooling. Tale tipo di raffreddamento è realizzato immettendo i gas di scarico sulle pareti dell'ugello attraverso un collettore che ne abbraccia l'intera circonferenza.

L'estensione dell'ugello è realizzata da due pareti in lega di nickel intervallate da bande circolari in CRES: tale costruzione saldata conferisce ottima resistenza termica alle pareti e una buona resistenza agli sforzi radiali a cui l'ugello è sottoposto.

Di seguito sono confrontati i principali parametri del motore con e senza l'espansione dell'ugello, ricavati tramite il software RPA:

\begin{table}[H]

\centering
\begin{tabular}{|c|c|c|c|c|c|c|}
\hline
& $\bm{p_e \, [bar]}$ & $\bm{T_e \, [K]}$ & $\bm{H \, [kJ/kg]}$ & $\bm{\gamma}$ & $\bm{\rho \, [kg/m^3]}$ & $\bm{v_e \, [m/s]}$ \\
\hline
\textbf{10:1} & $0.803$ & $1673.7$ & $-5058.1$ & $1.2439$ & $0.1304$ & $2910.6$ \\
\hline
\textbf{16:1} & $0.423$ & $1473.0$ & $-5429.9$ & $1.2521$ & $0.0781$ & $3035.6$ \\
\hline
\end{tabular}

\vspace{5pt}

\begin{tabular}{|c|c|c|c|c|c|c|}
\hline
& $\bm{I_{vac} \, [s]}$ & $\bm{I_{opt} \, [s]}$ & $\bm{I_{sl} \, [s]}$ & $\bm{T_{vac} \, [kN]}$ & $\bm{T_{opt} \, [kN]}$ & $\bm{T_{sl} \, [kN]}$ \\
\hline
\textbf{10:1} & $297.36$ & $276.44$ & $270.95$ & $7816.3$ & $7266.3$ & $7122.2$ \\
\hline
\textbf{16:1} & $306.25$ & $288.59$ & $263.99$ & $8050.0$ & $7585.9$ & $6939.3$ \\
\hline
\end{tabular}

\caption{Confronto tra ugello 10:1 e 16:1}
\label{table:confronto_ugello}

\end{table}

Si può notare un miglioramento nella spinta e nell'impulso specifico in corrispondenza dell'espansione ottima e dell'espansione nel vuoto, mentre si ha un calo di prestazione a livello del mare: ciò è dovuto al fatto che l'ugello sovraespande in maniera più marcata a pressione standard, poiché il punto di espansione ottima viene spostato a pressioni inferiori. Ciò non risulta essere un problema in quanto la spinta rimane sufficiente al lancio, mentre i benefici ottenuti alle quote di missione sono rilevanti.

 \pagebreak