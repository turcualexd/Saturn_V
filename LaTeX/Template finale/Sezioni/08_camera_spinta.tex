\section{Camera di spinta}
\label{sec:camera spinta}

\subsection{Descrizione del sistema}
\label{subsec:descrizione camera spinta}

L’intero gruppo della camera di spinta è costituito dal LOX dome, dal piatto di iniezione e dal corpo della camera di spinta, ovvero dalla camera di combustione per bruciare propellenti seguita da un ugello di espansione a forma di campana, necessario ad espellere a velocità elevata i gas prodotti dai propellenti bruciati e poter così generare la spinta.

La camera di spinta è di circa 3.35 metri di lunghezza e 2.74 metri di diametro all'estremità inferiore dell'ugello. Il suo corpo è formato da tubi fino al piano in cui il rapporto di espansione diventa pari a 10:1; in questi scorre il 70\% del carburante, fornendo un raffreddamento rigenerativo ed evitando che il materiale del tubo si sciolga durante il funzionamento del motore.

I tubi che compongono la camera di spinta sono costruiti in Inconel X-750, una lega a base di nichel resistente alle alte temperature e trattabile termicamente, scelta come materiale poiché forniva gli elevati rapporti forza-peso necessari per resistere ai requisiti di spinta del motore; inoltre, l'elevata resistenza di questa lega ha consentito la progettazione di tubi con sezioni di parete più sottili, con conseguente diminuzione del peso.

Il corpo sopra il piano corrispondente a rapporto di espansione 3:1 (circa 76.2 cm sotto il piano di gola) è costituito da 178 tubi primari con diametro esterno di 2.78 cm, mentre dal piano con rapporto di espansione 3:1 al piano a 10:1 si biforcano diventando 356 tubi secondari con diametro esterno di 2.54 cm.

Questa biforcazione è dovuta principalmente alla geometria dell'ugello e alle proprietà fisiche del materiale usato. La circonferenza di un ugello è minima in corrispondenza della gola mentre aumenta nella sezione di espansione: l'entità dell'aumento di circonferenza ottenibile con un numero fisso di tubi è quindi limitata da quanto questi ultimi possano essere lavorati per rastremazione. Quando si raggiunge il punto in cui la circonferenza non può essere ulteriormente aumentata con il numero prefissato di tubi si rende quindi necessario un giunto di biforcazione.

I gas di scarico della turbina, dopo essere passati attraverso lo scambiatore di calore, vengono convogliati al collettore di scarico della turbina, la cui funzione è quella di raccogliere e distribuire uniformemente il gas di scarico tra le pareti dell’estensione dell'ugello, che altrimenti non sarebbe raffreddata.  \cite{enginehistory} \cite{heroicrelics}


\subsection{Piatto d'iniezione}
\label{subsec:piatto iniezione}

\rfig{iniettore}{Piatto di iniezione}{iniettore}{0.3}

Gli iniettori sono collocati all’estremo superiore della camera di spinta e hanno lo scopo di distribuire il propellente in camera, regolando il rapporto di diluizione, la pressione e lo schema di spruzzo al fine di avviare e sostenere una combustione stabile. Per determinare questi valori sono stati necessari circa 3200 test su larga scala: al fine di generare un’esplosione controllata, risulta fondamentale che essa sia dinamicamente stabile, ossia che sia prevedibile e non crei punti caldi che porterebbero alla fusione di componenti del motore.

La faccia del piatto di iniezione, realizzata in CRES e strutturata in 31 anelli, conta 1428 orifizi per l’ossidante e 1404 orifizi per il carburante. I getti vengono atomizzati attraverso una disposizione a doppietti omogenei, i vapori di combustibile e di ossidante si miscelano e reagiscono a formare i gas propellenti, destinati successivamente all’espansione in ugello. La disposizione a doppietti omogenei è vantaggiosa rispetto ad altre: è di facile realizzazione, risulta stabile e affidabile e genera una buona miscelazione dei propellenti, a costo di necessitare di una camera più lunga rispetto ad altre tecniche di atomizzazione.

Gli anelli per il carburante sono alimentati attraverso un collettore radiale, mentre gli anelli per l’ossidante sono alimentati dal LOX dome tramite fori assiali.
Il collettore incorpora due ingressi per il montaggio delle valvole di ossidante e una flangia per la linea di alimentazione dell’ossidante allo scambiatore di calore. Per evitare vorticità nell’ossidante, il collettore è isolato in due compartimenti da due argini toroidali. Solamente il 30\% del combustibile viene indirizzato direttamente al collettore, mentre il restante 70\% viene utilizzato per il raffreddamento rigenerativo della camera di spinta.
Sono poi presenti due alloggiamenti per gli ignitori del combustibile in ciascuno dei 12 scomparti esterni, e un alloggiamento del combustibile nel compartimento centrale, tutti collegati al collettore da singoli tubi di alimentazione.

La stabilità della combustione è raggiunta principalmente mediante l’uso dei deflettori (baffles), oltre che variando l’angolo di impingement e il diametro degli orifizi in funzione della posizione sul piatto d’iniezione.

I deflettori in particolare alterano le caratteristiche acustiche di risonanza della camera di combustione, smorzando così le onde d’urto generate dalla combustione. I 12 deflettori radiali in rame sono alimentati dal deflettore circolare esterno. La configurazione dei deflettori utilizzata per il propulsore è stata ottenuta a seguito di vari test, nel quale si è ricercata la maggior stabilità di combustione possibile. Nella configurazione finale, i deflettori misurano circa 8 cm ciascuno e sono tutti dump-cooled, ovvero il raffreddamento è realizzato attraverso la circolazione di carburante all’interno del deflettore che viene successivamente scaricato nella camera di combustione. \cite{f-1_manual}\cite{JPP}

\begin{table}[H]

\centering
\begin{tabular}{|c|c|c|c|c|c|}
\hline
& $\bm{\dot{m} \, [kg/s]}$ & $\bm{A \, [m^2]}$ & $\bm{\Delta p \, [bar]}$ & $\bm{\rho \, [kg/m^3]}$ & $\bm{v \, [m/s]}$ \\
\hline
$\bm{fuel}$ & $742.09$ & $0.05484$ & $641$ & $810$ & $17.07$ \\
\hline
$\bm{oxidizer}$ & $1788.97$ & $0.03968$ & $2100$ & $1141$ & $40.54$ \\
\hline
\end{tabular}

\caption{Dati reali del piatto d'iniezione \cite{f-1_manual}\cite{JPP}}
\label{table:piatto iniezione}

\end{table}

A partire dai dati in \autoref{table:piatto iniezione}, è possibile inoltre stimare la velocità media teorica dei due propellenti all'uscita dai rispettivi iniettori:

\begin{empheq}{gather*}
	C_{D,f} = \frac{\dot{m}_{f}}{A_{f} \sqrt{2 \, \Delta p_{f} \, \rho_{f}}} = 13.28
	\qquad
	C_{D,ox} = \frac{\dot{m}_{ox}}{A_{ox} \sqrt{2 \, \Delta p_{ox} \, \rho_{ox}}} = 20.59
	\\
	v_{f} = C_{D,f} \sqrt{\frac{2 \, \Delta p_{f}}{\rho_{f}}} = 16.71 \, \text{m/s}
	\qquad
	v_{ox} = C_{D,ox} \sqrt{\frac{2 \, \Delta p_{ox}}{\rho_{ox}}} = 39.51 \, \text{m/s}
\end{empheq}

Tali risultati sono paragonabili alle velocità reali: la $v_{f}$ si discosta del 2.11\% dal valore reale, mentre la $v_{ox}$ si discosta del 2.54\%.

\subsection{Camera di combusione}
\label{subsec:camera_di_combustione}

\lfig{camera_comb}{Camera di combustione}{camera_comb}{0.3}

\vspace{3pt}

Nel motore F-1 è presente una camera di combustione cilindrica con una parte finale convergente che termina con la sezione di gola.
La camera di combustione funge da involucro che deve mantiene i propellenti per un periodo sufficiente a garantire la completa miscelazione e combustione. Il tempo di permanenza richiesto, o tempo di residenza, è una funzione di molti parametri, ovvero combinazione di propellenti, le condizioni di iniezione e la geometria del combustore (rapporto di contrazione, numero di Mach, livello di turbolenza).

Un parametro utile relativo al volume della camera e il tempo di residenza è la "lunghezza caratteristica", L*, ossia il volume della camera diviso l’area di gola: \cite{AIAA_book_1}

\begin{empheq}{gather*}
            L^* = \frac{V_{cc}}{A_{t}}                                                                       
\end{empheq}

Il concetto di L* è molto più facile da visualizzare rispetto al più elusivo "tempo di residenza", espresso in piccole frazioni di secondo, infatti si tratta di un sostituto per determinare il tempo di permanenza nella camera dei propellenti.

Un altro parametro fondamentale per il calcolo del tempo di residenza è la velocità caratteristica c*.
Il valore c* aumenta con L* fino a un massimo asintotico, ma l’aumento di L* oltre un certo punto tende a diminuire le prestazioni complessive del motore a causa di quanto segue: un maggiore L* si traduce in maggiore volume e peso della camera di spinta, con conseguente aumento della superficie che ha bisogno di raffreddamento e aumento delle perdite termiche e dovute all’attrito.
Il metodo abituale per stabilire la L* di un nuovo progetto della camera di spinta si basa in gran parte sull'esperienza passata con propellenti e dimensioni del motore simili. 
Nel caso della coppia di LOX/RP-1 la L* è compresa tra i valori 1÷1.30 metri, e per motivi progettuali descritti in precedenza è stata fissata la misura di L* a 1 m.
Partendo da questo dato e dalla dimensione dell’area di gola del motore è possibile modellare la camera di combustione.

Invertendo la formula della L* è possibile ottenere il volume della camera di combustione $V_{cc}$ che comprende, oltre alla parte cilindrica, anche la parte convergente.

\begin{empheq}{gather*}
          V_{cc} = {L^* A_{t}}                                                                       
\end{empheq}

È possibile anche calcolare la velocità caratteristica c* e il tempo di residenza attraverso le seguenti due formule:


\begin{empheq}{gather*}
             c^* =  \frac {p_{cc}  A_{t}} {\Gamma\ { \frac {p_{cc}}{\sqrt{{\frac {R}{MM}} T_{cc}}}  A_{t} }}          \qquad 
          \Gamma\ = \sqrt{\gamma\ (\frac{2}{\gamma\ +1})^{\frac {\gamma\ +1}{\gamma\ -1}}}                 \qquad
            t_{r} = \frac {L^*}{c^*}
\end{empheq}


Per calcolare le dimensioni reali di entrambe le parti della camera di combustione, ovvero la parte cilindrica e quella del convergente, è necessario fissare due parametri al fine dello sviluppo del modello.
Si tratta del rateo di contrazione, ossia il rapporto tra la sezione della camera cilindrica e la sezione di gola, e l’angolo di inclinazione del convergente


\begin{empheq}{gather*}
          \epsilon\_{cc} = \frac {A_{cil}}{A_{t}} = 1.307     \qquad     %indicare fonte dato  
           \theta\ = 13^{\circ}                                                             
\end{empheq}

Sfruttando la trigonometria si ottiene la lunghezza assiale del tratto convergente e considerando quest’ultimo come una figura tronco conica si calcola il volume. Partendo da quanto calcolato è possibile ottenere le dimensioni del tratto cilindrico:


\begin{empheq}{gather*}
         A_{cil} = \epsilon\_{cc} A_{t}                       \qquad
         R_{cil} = \sqrt{\frac{A_{cil}}{\pi}}            \qquad      
         a = {R_{cil} - R_{t}}                                  \\                          
\end{empheq}

\begin{empheq}{gather*}
         L_{conv} = \frac {a}{\tan(\theta\ )}             \qquad
         V_{conv} = \frac {(A_{cil} + A_{t} + \sqrt{A_{cil} A_{t}}) L_{conv}}{3}    \qquad                           
         L_{cil} = \frac {V_{cc} - V_{conv}}{A_{cil}} \qquad
         V_{cil} = L_{cil} A_{cil}                              
\end{empheq}

\begin{table}[H]

\centering
\begin{tabular}{|c|c|c|c|c|c|c|c|}
\hline
$\bm{A_{t} \, [m^2]}$ & $\bm{A_{cil} \, [m^2]}$ & $\bm{L_{conv} \, [m]}$ &  $\bm{L_{cil} \, [m]}$ & $\bm{V_{conv} \, [m^3]}$ & $\bm{V_{cil} \, [m^3]}$ & $\bm{V_{cc} \, [m^3]}$ & $\bm{A_{tot_{int}} \, [m^2]}$\\
\hline
$0.6207$ & $ 0.8113$ & $0.2758$ &  $0.5224$ & $0.1969$ & $0.4238$ & $0.6207$ & $2.5152$ \\
\hline
\end{tabular}


\caption{Tabella riassuntiva camera di combustione}
\label{table:geometria_cc}
\end{table}



\subsection{Modellazione dell'ugello}
\label{subsec:modellazione ugello}

L’obiettivo principale che si persegue nella progettazione dell’ugello di un endoreattore è quello di ottenere una forma che minimizzi le perdite di spinta per un qualsiasi rapporto di espansione richiesto.

Si procede quindi ad illustrare il metodo ideato da Rao per una progettazione ottimale rispetto ad un ugello di forma tronco-conica, con lo stesso rapporto di espansione, preso da riferimento:

\begin{empheq}{equation*}
L_{conico} = \frac{\left( \sqrt{\epsilon} - 1 \right) - R_t}{\tan {15\degree}}
\end{empheq}
\vspace{5pt}

dove $ R_t $ indica il raggio di gola dell’ugello e 15° è l’angolo standard di semi-apertura dell’ugello.

\vspace{5mm}

\rfig{ugello_TOP}{Definizioni geometriche}{ugello_TOP}{0.5}

La forma a campana ottimale può essere approssimata da una parabola inclinata grazie a considerazioni geometriche, permettendo anche di abbozzare velocemente una forma dell’ugello che contempla una perdita di prestazioni trascurabile a livello di spinta. Proprio per questo motivo, questa tipologia è anche chiamata ugello TOP (Thrust Optimized Parabolic) ed ha effettivamente trovato applicazione pratica nei vettori di lancio perché ha performance migliori quando sovra-espande a livello del mare (le pareti dell’ugello TOP aiutano a ritardare la separazione del flusso grazie ad un’elevata contropressione) rispetto ad un ugello ottimizzato perfettamente a campana. Inoltre, la forma dell’ugello varia in modo minimale in base ai propellenti usati e perciò una stessa famiglia di ugelli TOP può essere adattata per qualsiasi combinazione di ossidante e combustibile.

I parametri di partenza sono: il rapporto di espansione $ \epsilon $, il raggio di gola $ R_t $ e la percentuale di campana $ \%_{bell} $ che si vuole ottenere; quest’ultimo valore deve essere compreso tra il valore massimo di 85\%, a cui si raggiunge un livello di efficienza dell’ugello del 99\% e che può essere aumentato solo di un ulteriore 0.2\% con una percentuale di campana 100\%, e il valore minimo del 70\%, a cui si comincia ad ottenere un notevole degrado di prestazioni. Si ricavano quindi a cascata:

\vspace{5pt}
\begin{empheq}{alignat*=2}
& R_e = \sqrt{\epsilon} \, R_t		&\qquad		& \text{raggio della sezione d’efflusso}
\\
& L_{ugello} = \%_{bell} \frac{\left( \sqrt{\epsilon} - 1 \right) - R_t}{\tan {15\degree}}
&\qquad		& \text{lunghezza dell’ugello}
\end{empheq}
\vspace{5pt}

Si ricavano poi gli angoli $ \theta_n $, riferito al punto di inflessione N, e $ \theta_e $, riferito alla sezione d’uscita, per interpolazione grafica da curve analitiche ottenute sperimentalmente per determinati valori di $ \%_{bell} $ (grafico in \autoref{fig:angoli_bell}).

La prima parte di modellazione vera e propria consiste nella costruzione della gola dell’ugello secondo una geometria ottimale usata da Rao (ai tempi ingegnere alla Rocketdyne) e basata sull’intersezione di due archi di circonferenza definiti come segue:

\begin{empheq}{equation*}
x = 1.5 \, R_t \cos \theta	\qquad	y = R_t \left( 1.5 \, \sin \theta + 1.5 + 1 \right)
\end{empheq}

per la sezione di entrata, con $ -103\degree < \theta < -90\degree $ (l’angolo iniziale di -103° è scelto dal progettatore della camera di combustione \cite{nozzle_design} ma può anche essere fissato ad un valore differente);
\vspace{5pt}

\begin{empheq}{equation*}
x = 0.382 \, R_t \cos \theta	\qquad	y = R_t \left( 0.382 \, \sin \theta + 0.382 + 1 \right)
\end{empheq}

per la sezione di uscita, con $ -90\degree < \theta < \theta_n - 90\degree $.
\vspace{5mm}

Per la costruzione della campana è invece necessario definire prima tre punti geometrici:

\begin{itemize}[wide,itemsep=8pt,topsep=8pt]

\item
punto di inflessione N: $ \quad \text{N} = \begin{bmatrix} N_x \\ N_y \end{bmatrix} = \begin{bmatrix}
0.382 \, R_t \cos \left( \theta_n - 90\degree \right) \\
R_t \left[ 0.382 \, \sin \left( \theta_n - 90\degree \right) + 0.382 + 1 \right]
\end{bmatrix} $
\item
punto tangente alla sezione d'efflusso E: $ \quad \text{E} = \begin{bmatrix} E_x \\ E_y \end{bmatrix} = \begin{bmatrix} R_e \\ L_{ugello} \end{bmatrix} $
\item
punto Q di intersezione delle rette passanti da N con inclinazione $ \theta_n $ e da E con inclinazione $ \theta_n $:

\begin{empheq}{alignat*=2}
&\overrightarrow{NQ} = m_1 x + C_1 \; \text{con} \; m_1 = \tan \theta_n \; \text{e} \; C_1 = N_y - m_1 N_x
&\qquad
& Q_x = \frac{C_2 - C_1}{m_1 - m_2}
\\
&\overrightarrow{QE} = m_2 x + C_2 \; \text{con} \; m_2 = \tan \theta_e \; \text{e} \; C_2 = E_y - m_2 E_x
&\qquad
& Q_y = \frac{m_1 C_2 - m_2 C_1}{m_1 - m_2}
\end{empheq}

\end{itemize}
\vspace{5pt}

La campana infine risulta essere una curva di Bézier quadratica di equazione:

\begin{empheq}{alignat*=2}
& x(t) = \left( 1 - t \right)^2 N_x + 2 \left( 1 - t \right) t \, Q_x + t^2 E_x &\qquad
& 0 \le t \le 1 \\
& y(t) = \left( 1 - t \right)^2 N_y + 2 \left( 1 - t \right) t \, Q_y + t^2 E_y &\qquad
& 0 \le t \le 1
\end{empheq}

\subsection{Confronto tra ugello 10:1 e 16:1}
\label{subsec:confronto ugello}

I motori F-1 prodotti dalla Rocketdyne avevano in origine un ugello il cui rapporto di espansione era 10:1; essi infatti non furono inizialmente progettati nello specifico per lo stadio di lancio del Saturn V. Gli ingegneri decisero quindi a posteriori di aggiungere un'espansione dell'ugello iniziale allo scopo di migliorare vari parametri del lanciatore: i più rilevanti sono l'impulso specifico nel vuoto e la quota a cui è raggiungibile l'espansione ottima (adattata alla traiettoria che il lanciatore avrebbe percorso).

Tale espansione non poteva essere raffreddata dal già presente regenerative cooling: si optò dunque per una soluzione che prevedesse l'utilizzo dei gas di scarico della turbina, ricchi di carbonio e quindi con bassa conducibilità termica, per il raffreddamento attraverso film cooling. Tale tipo di raffreddamento è realizzato immettendo i gas di scarico sulle pareti dell'ugello attraverso un collettore che ne abbraccia l'intera circonferenza.

L'estensione dell'ugello è realizzata da due pareti in lega di nickel intervallate da bande circolari in CRES: tale costruzione saldata conferisce ottima resistenza termica alle pareti e una buona resistenza agli sforzi radiali a cui l'ugello è sottoposto.

Di seguito sono confrontati i principali parametri del motore con e senza l'espansione dell'ugello, ricavati tramite il software RPA:

\begin{table}[H]

\centering
\begin{tabular}{|c|c|c|c|c|c|c|}
\hline
& $\bm{p_e \, [bar]}$ & $\bm{T_e \, [K]}$ & $\bm{H \, [kJ/kg]}$ & $\bm{\gamma}$ & $\bm{\rho \, [kg/m^3]}$ & $\bm{v_e \, [m/s]}$ \\
\hline
\textbf{10:1} & $0.803$ & $1673.7$ & $-5058.1$ & $1.2439$ & $0.1304$ & $2910.6$ \\
\hline
\textbf{16:1} & $0.423$ & $1473.0$ & $-5429.9$ & $1.2521$ & $0.0781$ & $3035.6$ \\
\hline
\end{tabular}

\vspace{5pt}

\begin{tabular}{|c|c|c|c|c|c|c|}
\hline
& $\bm{I_{vac} \, [s]}$ & $\bm{I_{opt} \, [s]}$ & $\bm{I_{sl} \, [s]}$ & $\bm{T_{vac} \, [kN]}$ & $\bm{T_{opt} \, [kN]}$ & $\bm{T_{sl} \, [kN]}$ \\
\hline
\textbf{10:1} & $297.36$ & $276.44$ & $270.95$ & $7816.3$ & $7266.3$ & $7122.2$ \\
\hline
\textbf{16:1} & $306.25$ & $288.59$ & $263.99$ & $8050.0$ & $7585.9$ & $6939.3$ \\
\hline
\end{tabular}

\caption{Confronto tra ugello 10:1 e 16:1}
\label{table:confronto_ugello}

\end{table}

Si può notare un miglioramento nella spinta e nell'impulso specifico in corrispondenza dell'espansione ottima e dell'espansione nel vuoto, mentre si ha un calo di prestazione a livello del mare: ciò è dovuto al fatto che l'ugello sovraespande in maniera più marcata a pressione standard, poiché il punto di espansione ottima viene spostato a pressioni inferiori. Ciò non risulta essere un problema in quanto la spinta rimane sufficiente al lancio, mentre i benefici ottenuti alle quote di missione sono rilevanti.