\section{Prodotti gas generator analizzati con software NASA CEA}
\label{appendix:prodotti_gas_generator}
In questa appendice sono presentati i dati ottenuti tramite simulazione in NASA CEA (CEAM) per quanto riguarda la combustione che avviene nel Gas Generator. Il problema risolto dal CEAM è di tipo \textit{hp}, per cui la combustione raggiunge l'equilibrio senza l'imposizione di una particolare temperatura (l'assegnazione di una temperatura fissata nei problemi di tipo \textit{tp} può bloccare l'equilibrio e imporre quindi una temperatura, processo che non è di nostro interesse). I dati in input sono i seguenti:
\\
\begin{table}[H]

\centering
\begin{tabular}{|c|c|c|c|c|}
\hline
$\bm{Prob}$ & $\bm{p_c \, [bar]}$ & $\bm{T_{RP1} \, [K]}$ & $\bm{T_{LOX} \, [K]}$ & $\bm{O/F}$ \\
\hline
$hp$ & $67.57$ & $293.15$ & $90.15$ & $0.416$ \\
\hline
\end{tabular}
\caption{Dati in input CEAM}
\label{table:input_CEAM}
\end{table}
In output sono stati ottenuti i seguenti dati (sono stati selezionati alcuni valori rilevanti)
\begin{table}[H]

\centering
\begin{tabular}{|c|c|c|}
\hline
$\bm{T_{comb} \, [K]}$ & $\bm{MW \, [g/mol]}$ & $\bm{c_p \, [Kj/KgK]}$ \\
\hline
$1243$ & $19.48$ & $9.8082$\\
\hline
\end{tabular}

\caption{Dati in output CEAM}
\label{table:output_CEAM}

\end{table}
La temperatura di combustione ottenuta $T_{comb}$ con l'esecuzione del programma è più alta rispetto a quella tabulata nei manuali del motore \cite{f-1_manual}, che è di 1062 K. Alcune ipotesi che possono in parte giustificare questa differenza sono:
\begin{itemize}
\item Idealità dell'ambiente di combustione nel caso dell'analisi CEAM, in quanto la combustione arriva perfettamente all'equilibrio chimiconel problema \textit{hp} senza considerare la vera geometria della camera di combustione del GG.
\item Temperature non uniformi nel caso reale poichè l'o/f di iniezione non è uniforme, in quanto si hanno zone più ricche di fuel sulle zone esterne del piatto di iniezione (ovvero zone raffreddate). 
\end{itemize}

In seguito viene presentata la percentuale in massa dei prodotti di combustione, ottenuti tramite la stessa esecuzione del problema \textit{hp} del GG eseguita da CEAM:

\begin{table}[H]

\centering
\begin{tabular}{|c|c|c|c|c|c|c|c|}
\hline
$\bm{CH_4}$ & $\bm{CO}$ & $\bm{CO_2}$ & $\bm{C_2H_4}$ & $\bm{C_2H_6}$ & $\bm{H_2}$ & $\bm{H_2O}$ & $\bm{C(gr)}$ \\
\hline
$15.22 \%$ & $28.36 \%$ & $6.27 \%$ & $0.003 \%$ & $0.011 \%$ & $5.018 \%$ & $9.705 \%$ & $35.41 \%$ \\
\hline
\end{tabular}

\caption{Percentuali in massa dei prodotti di combustione (CEAM)}
\label{table:output_CEAM_percentage}

\end{table}