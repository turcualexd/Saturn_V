\section{Serbatoi e pressurizzazione}
\label{sec:serbatoi e pressurizzazione}

\subsection{Serbatoi: descrizione e dimensionamento}
\label{subsec:Serbatoi:descrizione_e_dimensionamento}

\rfig{tank}{Raffigurazione serbatoi}{tank}{0.4}

Il primo stadio ha un diametro di circa 10 m e un’altezza di 42 m. La maggior parte del volume interno alla struttura è occupato dai serbatoi, ovvero il serbatoio di LOX, posto più in alto, e quello dell’RP1. 

l serbatoio dell’ossidante ha una forma cilindrica, chiusa alle due estremità da paratie ellissoidali, per un’altezza totale di 19.53 m, mentre il diametro è approssimabile al diametro della struttura interna, ossia 10.06 m e il volume di ossigeno liquido contenuto è 1266220 litri. 
La struttura si compone di una skin in lega 2219-T87, una lega di alluminio con rame come legante principale che si presta all’utilizzo grazie alle alte proprietà meccaniche in un ampio range di temperatura, dai -270°C ai 300°C circa, nel quale ricade la temperatura di stoccaggio del LOX; è inoltre un materiale con buona saldabilità, elevata tenacità a frattura e alta resistenza a cricche da corrosione e da sforzo. 
Dal serbatoio di ossidante partono 5 linee di aspirazione isolate di diametro 43,18 cm passanti attraverso il serbatoio di combustibile fino ai motori. All’estremità delle linee di aspirazione sono collocate valvole di emergenza e soffietti flessibili in grado di attenuare le vibrazioni e adattarsi a espansioni e contrazioni legate alla variazione di temperatura. 
Il serbatoio termina con un dispositivo centrale e paratie cruciformi che hanno lo scopo di evitare la formazione di vortici man mano che l’ossidante defluisce.
All’interno della struttura del serbatoio di LOX, ancorate ai deflettori ad anello, vi sono quattro bombole di elio ad alta pressione dal volume di 0.88 m3 ciascuna, necessarie alla pressurizzazione.

Il collegamento tra i due serbatoi è rappresentato dall’intertank, una struttura composta da cinque telai e pannelli ondulati in lega di alluminio 7075-T6 ad altissima resistenza meccanica.

Il serbatoio di combustibile ha geometria analoga al serbatoio di ossidante, con un’altezza di 13.13 m e un volume di 791151 litri. La struttura è anch’essa costituita da una skin di lega di alluminio 2219-T87.

Di fondamentale importanza per lo studio dei serbatoi è il dimensionamento dello spessore delle pareti, poiché è necessario ottimizzare il peso senza però mette a rischio il funzionamento della struttura.
I carichi strutturali ai quali è sottoposta la parete sono diversi, tra cui quelli dovuti alle pressioni interne, carichi assiali di spinta, carichi aerodinamici e termodinamici e quelli prodotti dalla disposizione di montaggio.
È possibile procedere con il dimensionamento delle pareti sviluppando una serie di calcoli partendo da valori di progetto, come il volume dei propellenti necessari al funzionamento, l’ingombro esterno dei singoli serbatoi e considerando noto il materiale utilizzato.
Il volume totale all’interno del serbatoio non è composto unicamente dal propellente, ma ci sono anche degli altri volumi da considerare come il volume vuoto necessario alla pressurizzazione.

%calcoli e tabelle con procedimenti e dati e risultati e cazzi e mazzi

Come primo passo si calcolano i volumi fondamentali dei serbatoi, ovvero quelli della parte cilindrica e delle due cupole ellissoidali, conoscendone il raggio e le altezze, per poi calcolare il volume totale interno del serbatoio considerando anche il volume occupato dalle bombole di elio nel serbaotio del LOX e dalle cinque linee di aspirazione passanti attraverso il serbatoio dell'RP-1:

\begin{empheq}{gather*}
            Vol_{e} =  \frac{2 \pi\ a^2 b}{3}                                  \qquad
            Vol_{c} = {\pi\ a^2 lc}                                                 \\
            Vol_{He} = {4 Vol_{1b}}                              \qquad
            Vol_{tubi} =  {5 \pi\ r_{t}^2 l_{t}}                                                                 \\
\vspace{1pt}                     \\
            V_{totRP-1} = {2 Vol_{e} + Vol_{c} - Vol_{t}}              \qquad
            V_{totLOX} =  {2 Vol_{e} + Vol_{c} - Vol_{He}}
\end{empheq}

\vspace{5pt}

Noto il volume totale occupato dal propellente liquido si ricavano le dimensioni della parte del serbatoio necessaria alla pressurizzazione, ovvero il volume vuoto, e di conseguneza l'altezza del propellente:
%formuleeeee

\begin{empheq}{gather*}
            Vol_{u} = {V_{tot} - V_{prop}}                                \\
            H_{prop} = {lc + 2b - H_{u}}
\end{empheq}

\vspace{5pt}

La pressione nel serbatoio non è nota, ma va calcolata, mentre quella del volume vuoto è un dato di progetto del motore.
Con quanto ricavato in precedenza si calcolano le pressioni interne dei serbatoi e le pressioni totali

\begin{empheq}{gather*}
            P_{i} = {\rho\ g H}                             \\
            P_{tot} = { P_{i} + P_{u}}                                              
\end{empheq}

È ora possibile procedere al calcolo degli spessori introducendo alcuni parametri come il fattore di stress K e il rateo dell’ellisse k, cioè il rapporto tra l'asse maggiore e l'asse minore

\begin{empheq}{gather*}
            K = {0.8}                                  \qquad
            k = \frac{a}{b}                                                 \qquad
            R = {a k}                                                              \\\
            t_{k} = \frac{K P_{tot} a}{\Sigma_{y}}                                                        \qquad       
            t_{cr} =  \frac{P_{tot} R}{2\Sigma_{y}}                                                                   \qquad
            t_{e} = {t_{k} + t_{cr}}                                                        \qquad
            t_{c} = \frac{P_{tot} a}{\Sigma_{y}} 
\end{empheq}



\begin{table}[H]

\centering
\begin{tabular}{|c|c|c|c|c|c|c|c|}
\hline
& $\bm{\rho \, [kg/m^3]}$ & $\bm{\sigma_{r} \, [psi]}$ & $\bm{\sigma_{y} \, [psi]}$ & $\bm{E \, [psi]}$ & $\bm{\nu\, [-]}$& $\bm{\Sigma_{r}\, [-]}$ & $\bm{\Sigma_{y}\, [-]}$ \\
\hline
\textbf{Al2219-T87} & $0.103$ & $69000$ & $57000$ & $10.6e6$ & $0.33$ & $\frac{\sigma_{r}}{1.3}$ & $\frac{\sigma_{y}}{1.25}$ \\
\hline
\end{tabular}

\caption{Lega Alluminio 2219-T9}
\label{table:dati_materiale}

\vspace{5pt}

\begin{tabular}{|c|c|c|c|}
\hline
& $\bm{a \, [inch]}$ & $\bm{b \, [inch]}$ & $\bm{lc \, [inch]}$ \\
\hline
\textbf{RP-1} & $198$ & $120$ & $276$ \\
\hline
\textbf{LOX} & $198$ & $120$ & $528$ \\
\hline
\end{tabular}


\caption{Dati geometrici}
\label{table:dati_geometrici}

\vspace{5pt}

\begin{tabular}{|c|c|c|c|c|c|}
\hline
& $\bm{p_{u} \, [psi]}$ & $\bm{p_{i} \, [psi]}$ & $\bm{p_{tot} \, [psi]}$ & $\bm{p_{cre} \, [psi]}$ & $\bm{p_{crc} \, [psi]}$\\
\hline
\textbf{RP-1} & $25.2$ & $13.38$ & $38.58$ & $0.368$ & $0.1393$\\
\hline
\textbf{LOX} & $23$ & $27.92$ & $50.92$ & $0.641$ & $0.1457$\\
\hline
\end{tabular}


\caption{Tabella riassuntiva pressioni}
\label{table:pressioni}

\vspace{5pt}

\begin{tabular}{|c|c|c|c|c|c|c|}
\hline
& $\bm{Vol_{prop} \, [m^3]}$ & $\bm{Vol_{c} \, [m^3]}$ & $\bm{Vol_{e} \, [m^3]}$ & $\bm{Vol_{tubi/He} \, [m^3]}$ & $\bm{Vol_{tot} \, [m^3]}$ & $\bm{Vol_{u} \, [m^3]} $\\
\hline
\textbf{RP-1} & $ $ & $ $ & $ $ & $ $ & $ $ & $ $\\
\hline
\textbf{LOX} & $ $ & $ $ & $ $ & $ $ & $ $ & $ $\\
\hline
\end{tabular}


\caption{Tabella riassuntiva volumi}
\label{table:volumi}

\vspace{5pt}

\begin{tabular}{|c|c|c|c|c|}
\hline
& $\bm{t_{k} \, [m]}$ & $\bm{t_{cr} \, [m]}$ & $\bm{t_{e} \, [m]}$ & $\bm{t_{c}\, [m]}$\\
\hline
\textbf{RP-1} & $0.0034$ & $0.0035$ & $0.0035$ & $0.0043$\\
\hline
\textbf{LOX} & $0.0045$ & $0.0046$ & $0.0046$ & $0.0056$\\
\hline
\end{tabular}


\caption{Tabella riassuntiva spessori}
\label{table:spessori}


\end{table}



I valori ottenuti risultano essere gli spessori minimi; in realtà lo spessore dovrebbe essere leggermente maggiore per consentire la saldatura, l'instabilità e la concentrazione delle sollecitazioni. 

\subsection{Pressurizzazione e scambiatore di calore}
\label{subsec:pressurizzazione_e_scambiatore_di_calore}

La pressurizzazione del serbatoio di LOX è affidata al GOX, ossia ossigeno in forma gassosa. Una linea di rilevamento fornisce un feedback di pressione a valvole che controllano il flusso del GOX e permettono di mantenere la pressione di riempimento del serbatoio, man mano che l’ossidante defluisce, tra 18 e 20 psia durante le fasi di volo. La pressurizzazione dei serbatoi è un requisito necessario dall’avviamento del motore fino a fine missione per stabilire e mantenere un gradiente di pressione positivo.

Durante il volo la sorgente di pressurizzazione del serbatoio di combustibile è invece l’elio ad alta pressione. Tramite l’utilizzo di valvole viene mantenuta l’alta pressione all’interno del serbatoio mentre la pressione nelle cisterne di elio decresce. 
I condotti trasportano l’elio verso gli scambiatori di calore dei motori F-1 e dopo essere passato attraverso ad esso, i condotti di ritorno riportano l’elio gassoso riscaldato ed espanso nella parte superiore del serbatoio del carburante.

L’elio è stato scelto rispetto ad altre soluzioni, come l’azoto, grazie alla sua bassa densità (0.1785 kg/m3) che permette di risparmiare peso.
I serbatoi di elio sono posizionati all’interno del serbatoio di LOX perché grazie alla bassa temperatura dell’ambiente interno criogenico aumenta la densità dell’elio ed è possibile utilizzare serbatoi di alluminio più piccoli e leggeri con circa metà dello spessore delle pareti per immagazzinare una quantità maggiore di elio.
Similarmente i serbatoi di elio non possono essere posizionati all’interno dell’RP-1 poiché la bassa temperatura dell’elio potrebbe congelare il carburante; infatti, viene immagazzinato ad una temperatura criogenica di 90 K.
Le quattro “bottiglie” di elio sono lunghe circa 6 m, hanno un diametro di 56 cm e hanno un volume di 0.88 m3.

Per permettere la pressurizzazione nei serbatoi è necessario l’utilizzo di uno scambiatore di calore.
I serbatoi trasferiscono principalmente i propellenti alle turbopompe, per cui è necessaria la pressurizzazione per evitare il fenomeno di cavitazione all'ingresso della turbopompa. 
Per il motore F-1 si utilizza uno scambiatore di calore a fascio tubiero con 4 bobine e come fonte di calore per permettere lo scambio di calore è utilizzato il gas di scarico della turbina.
Tuttavia, è difficile trovare in letteratura metodi di prova appropriati o metodologie di progettazione per tali scambiatori di calore, poiché per la previsione accurata delle prestazioni, l'ottimizzazione del peso del veicolo di lancio e le prestazioni dello scambiatore di calore vengono valutate mediante test di prova. 
Per collegare lo scambiatore con i serbatoi sono presenti quattro condotti: due linee di andata verso lo scambiatore e due di ritorno verso i serbatoi.
L'elio fornito dalle bombole conservate nel serbatoio LOX viene indirizzato allo scambiatore di calore, espanso e quindi indirizzato al serbatoio dell’RP-1 tramite una linea di distribuzione situata all'estremità superiore del serbatoio per garantirne la pressurizzazione. 

Per il serbatoio del LOX si ha un sistema di pressurizzazione differente; infatti, il LOX ad alta pressione viene prelevato dal LOX dome, indirizzato allo scambiatore di calore, dove viene espanso in ossigeno gassoso (GOX) e in seguito diretto al serbatoio tramite la linea di distribuzione del GOX.

Le dimensioni esatte dello scambiatore di calore del motore F-1 dipendono dalla versione specifica del motore, poiché il motore F-1 è stato sottoposto a una serie di miglioramenti, tuttavia, in generale, ha una lunghezza di circa 1 metro e mezzo e un diametro che varia da 1 metro all'uscita della turbina a 60 cm al collettore di scarico della turbina. 


