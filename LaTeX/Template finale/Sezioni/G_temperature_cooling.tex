\section{Appendice ausiliaria capitolo 9}
\label{appendix:cooling_temp_definitions}

\subsection{Definizione temperature usate nel cooling}
\rfig{t_wg}{Definizione della temperatura muro - lato gas}{t_wg}{0.50}

La velocità del fluido all'esterno dello strato limite è la velocità del flusso libero e, attraversando lo strato limite perpendicolarmente alla parete, la velocità diminuisce fino ad annullarsi per soddisfare la condizione di aderenza. La temperatura a parete dovrebbe perciò essere pari alla temperatura di ristagno, ossia la temperatura raggiunta quando tutta l'energia cinetica viene trasformata in energia termica senza alcuna perdita. Nel caso di flussi molto veloci, l'aumento di temperatura è abbastanza elevato da provocare un processo di rallentamento viscoso non adiabatico. Per questo motivo, nell'ipotesi di parete adiabatica verso l'esterno, avviene un significativo scambio termico dal fluido in prossimità della parete, caratterizzato da bassa velocità e alta temperatura statica, verso il fluido più lontano dalla parete. A parete si avrà quindi una temperatura $T_{aw}$ più bassa della temperatura totale che caratterizza il flusso libero, mentre all'interno dello strato limite, affinché venga soddisfatta l'equazione dell'energia per flussi stazionari, deve essere necessariamente presente una regione in cui la temperatura è più alta di quella del flusso libero. Si delinea un andamento della temperatura come schematizzato in figura.
\vspace{20pt}

\subsection{Grafico della resistenza termica prodotta dal deposito carbonioso}
\rfig{r_d_graph}{Grafico deposito carbonioso in funzione di $\epsilon$}{r_d_graph}{0.4}
Il grafico di \autoref{fig:r_d_graph} rappresenta l'andamento della resistenza termica causata dal deposito carbonioso $R_d$ in funzione del rapporto di espansione $\epsilon$. Tale grafico è ricavato sperimentalmente da un endoreattore a propellente liquido LOX/RP-1 e rapporto di miscela $O/F$ pari a 2.35 e pressione camera di combustione $p_c$ di 1000 psia \cite{AIAA_book_2}. Tali valori ci permettono di concludere che, in prima approssimazione, il grafico possa essere sfruttato per ricavare il valore di $R_d$ del motore preso in esame. 
\\
\\
\\
\\
\\
\\
\\
\\
\\
\\
\\
\\
\\
\subsection{Dettagli sul sistema di introduzione dei gas combusti sulla parete interna dell'ugello aggiunto}
Le seguenti immagini mostrano come venga effettuato lo scarico dei gas combusti per il processo di film cooling. Fonti (\cite{site_exhaust} \cite{f-1_manual}
\twofig{exhaust_1}{Dettaglio 1}{exhaust_1}{exhaust_2}{Dettaglio 2}{exhaust_2}

\subsection{Comportamento del fluido RP1 in relazione alla pressione critica}

\rfig{satur}{Andamenti di T e flusso di calore, parametrizzati sul valore di pressione}{satur}{0.4}
 
Verranno analizzati due possibili scenari, rappresentati in figura: la curva $A_i$ descrive l'andamento del legame temperatura della parete – flusso di calore nel caso di pressione minore della pressione critica mentre la curva $B_i$ rappresenta l'andamento del legame nella condizione di pressione maggiore della pressione critica.

Studiando la curva A, il tratto A1-A2 rappresenta lo scambio di calore nelle condizioni in cui la temperatura della parete lato coolant non ha ancora raggiunto la temperatura di saturazione, in corrispondenza della pressione del refrigerante. Alla temperatura del punto A2, superata la temperatura di saturazione, il combustibile inizia a bollire, creando quindi delle “bolle” nella fascia a ridosso della parete. Queste crescono di dimensione nel flusso liquido piu freddo fino a che la velocità di condensazione del vapore supera la velocità di vaporizzazione: le bolle iniziano a collassare. Questo processo, che avviene ad alta frequenza, è detto “Nucleate boiling” (ebollizione nucleata). In corrispondenza di questo fenomeno il coefficiente di scambio termico aumenta, causando un aumento contenuto della temperatura a parete per un'ampia gamma di flussi di calore. Lo scambio di calore caratterizzato da ebollizione nucleata è rappresentato dal tratto A2-A3. Alla temperatura corrispondente al punto A3, un ulteriore aumento del flusso di calore porta ad un incremento di concentrazione di bolle tale per cui esse si combinano in un film di vapore a cui consegue una forte diminuzione del coefficiente del trasferimento del calore (tratto A3-A4). Lo scambio di calore raggiunto al punto A3 definisce il limite superiore del nucleate boiling, valore che viene quindi utilizzato come limite di progetto per il sistema di raffreddamento rigenerativo.
La curva B descrive le varie fasi del legame flusso di calore- temperatura della parete nel caso in cui la pressione sia al di sopra di quella critica: in queste condizioni di il fenomeno di nucleate boiling non si manifesta. Queste condizioni portano ad un aumento di temperatura proporzionale all'incremento del flusso di calore: in questo modo si raggiunge la temperatura limite per un valore di scambio di calore minore.

 