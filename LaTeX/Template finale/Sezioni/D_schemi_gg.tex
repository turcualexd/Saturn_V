\section{Schemi del gas generator}
\label{appendix:gg_schematics}

Per poter apprendere come l'O/F influisca sulla MM dei prodotti si è utilizzato il software NASA CEA. Si è utilizzato un problema HP perchè nell'analisi è importante capire anche la temperatura di equilibrio (abbiamo visto che il problema HP tendeva a sovrastimare la temperatura in camera, tuttavia come analisi preliminare, per l'attuale scopo ci interessa avere una linea guida generale). La temperatura di equilibrio in questo caso deve essere controllata attentamente poichè in uscita dal GG abbiamo il vincolo della palettatura di turbina. Si è imposto che la temperatura debba essere minore di 1500K sia in FR che OR. Si è fatto variare l'O/F da nel range $0.2 / 16$, si sono plottati i grafici di MM e T in funzione dell'O/F.