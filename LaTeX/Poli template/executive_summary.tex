% A LaTeX template for EXECUTIVE SUMMARY of the MSc Thesis submissions to 
% Politecnico di Milano (PoliMi) - School of Industrial and Information Engineering
%
% S. Bonetti, A. Gruttadauria, G. Mescolini, A. Zingaro
% e-mail: template-tesi-ingind@polimi.it
%
% Last Revision: October 2021
%
% Copyright 2021 Politecnico di Milano, Italy. NC-BY

\documentclass[11pt,a4paper,twocolumn]{article}

%------------------------------------------------------------------------------
%	REQUIRED PACKAGES AND  CONFIGURATIONS
%------------------------------------------------------------------------------
% PACKAGES FOR TITLES
\usepackage{titlesec}
\usepackage{color}

% PACKAGES FOR LANGUAGE AND FONT
\usepackage[utf8]{inputenc}
\usepackage[english]{babel}
\usepackage[T1]{fontenc} % Font encoding

% PACKAGES FOR IMAGES
\usepackage{graphicx}
\graphicspath{{Images/}} % Path for images' folder
\usepackage{eso-pic} % For the background picture on the title page
\usepackage{subfig} % Numbered and caption subfigures using \subfloat
\usepackage{caption} % Coloured captions
\usepackage{transparent}

% STANDARD MATH PACKAGES
\usepackage{amsmath}
\usepackage{amsthm}
\usepackage{bm}
\usepackage[overload]{empheq}  % For braced-style systems of equations

% PACKAGES FOR TABLES
\usepackage{tabularx}
\usepackage{longtable} % tables that can span several pages
\usepackage{colortbl}

% PACKAGES FOR ALGORITHMS (PSEUDO-CODE)
\usepackage{algorithm}
\usepackage{algorithmic}

% PACKAGES FOR REFERENCES & BIBLIOGRAPHY
\usepackage[colorlinks=true,linkcolor=black,anchorcolor=black,citecolor=black,filecolor=black,menucolor=black,runcolor=black,urlcolor=black]{hyperref} % Adds clickable links at references
\usepackage{cleveref}
\usepackage[square, numbers, sort&compress]{natbib} % Square brackets, citing references with numbers, citations sorted by appearance in the text and compressed
\bibliographystyle{plain} % You may use a different style adapted to your field

% PACKAGES FOR THE APPENDIX
\usepackage{appendix}

% PACKAGES FOR ITEMIZE & ENUMERATES 
\usepackage{enumitem}

% OTHER PACKAGES
\usepackage{amsthm,thmtools,xcolor} % Coloured "Theorem"
\usepackage{comment} % Comment part of code
\usepackage{fancyhdr} % Fancy headers and footers
\usepackage{lipsum} % Insert dummy text
\usepackage{tcolorbox} % Create coloured boxes (e.g. the one for the key-words)
\usepackage{stfloats} % Correct position of the tables

%-------------------------------------------------------------------------
%	NEW COMMANDS DEFINED
%-------------------------------------------------------------------------
% EXAMPLES OF NEW COMMANDS -> here you see how to define new commands
\newcommand{\bea}{\begin{eqnarray}} % Shortcut for equation arrays
\newcommand{\eea}{\end{eqnarray}}
\newcommand{\e}[1]{\times 10^{#1}}  % Powers of 10 notation
\newcommand{\mathbbm}[1]{\text{\usefont{U}{bbm}{m}{n}#1}} % From mathbbm.sty
\newcommand{\pdev}[2]{\frac{\partial#1}{\partial#2}}
% NB: you can also override some existing commands with the keyword \renewcommand

%----------------------------------------------------------------------------
%	ADD YOUR PACKAGES (be careful of package interaction)
%----------------------------------------------------------------------------


%----------------------------------------------------------------------------
%	ADD YOUR DEFINITIONS AND COMMANDS (be careful of existing commands)
%----------------------------------------------------------------------------


% Do not change Configuration_files/config.tex file unless you really know what you are doing. 
% This file ends the configuration procedures (e.g. customizing commands, definition of new commands)
\input{Configuration_files/config}

% Insert here the info that will be displayed into your Title page 
% -> title of your work
\renewcommand{\title}{Title of the thesis}
% -> author name and surname
\renewcommand{\author}{Name Surname}
% -> MSc course
\newcommand{\course}{Xxxxxxxxxxxx Engineering - Ingegneria Xxxxxxxxxxxx}
% -> advisor name and surname
\newcommand{\advisor}{Prof. Name Surname}
% IF AND ONLY IF you need to modify the co-supervisors you also have to modify the file Configuration_files/title_page.tex (ONLY where it is marked)
\newcommand{\firstcoadvisor}{Name Surname} % insert if any otherwise comment
%\newcommand{\secondcoadvisor}{Name Surname} % insert if any otherwise comment
% -> academic year
\newcommand{\YEAR}{20XX-20XX}

%-------------------------------------------------------------------------
%	BEGIN OF YOUR DOCUMENT
%-------------------------------------------------------------------------
\begin{document}

%-----------------------------------------------------------------------------
% TITLE PAGE
%-----------------------------------------------------------------------------
% Do not change Configuration_files/TitlePage.tex (Modify it IF AND ONLY IF you need to add or delete the Co-advisors)
% This file creates the Title Page of the document
% DO NOT REMOVE SPACES BETWEEN LINES!

\let\temp\newpage
\let\newpage\relax
\begin{titlepage}

\AddToShipoutPicture*{\BackgroundPic}

\hspace{-0.6cm}\includegraphics[width=0.6\textwidth]{logo_polimi_ing_indinf.eps}

\vspace{-0.2cm}
\Large{\textbf{\color{bluePoli}{\title}}}\\
\hspace*{\fill}

\vspace{-0.2cm}
\fontsize{0.3cm}{0.5cm}\selectfont \bfseries \textsc{\color{bluePoli} Laurea Triennale in \course}\\
\hspace*{\fill}

\vspace{-0.2cm}
\fontsize{0.3cm}{0.5cm} \selectfont \bfseries Autori: \textsc{\textbf{\authors}}\\
\hspace*{\fill}

\vspace{-0.4cm}
\fontsize{0.3cm}{0.5cm}\selectfont \bfseries Professore: \textsc{\textbf{\professor}}\\
\hspace*{\fill}

% if only ONE co-advisor is present:
%\vspace{-0.4cm}
%\fontsize{0.3cm}{0.5cm}\selectfont \bfseries Co-advisor: %\textsc{\textbf{\firstcoadvisor}}\\
% if more than one co-advisors are present:
%\vspace{-0.4cm}
%\fontsize{0.3cm}{0.5cm}\selectfont \bfseries Co-advisors: \textsc{\textbf{\firstcoadvisor}}\textsc{\textbf{\secondcoadvisor}}\\

\vspace{-0.4cm}
\fontsize{0.3cm}{0.5cm}\selectfont \bfseries Anno accademico: \textsc{\textbf{\YEAR}}

\small \normalfont

\vspace{11pt}

\centerline{\rule{1.0\textwidth}{0.4pt}}

\vspace{15pt}

\end{titlepage}
\let\newpage\temp

\thispagestyle{plain} % In order to not show the header in the first page

%%%%%%%%%%%%%%%%%%%%%%%%%%%%%%
%%     THESIS MAIN TEXT     %%
%%%%%%%%%%%%%%%%%%%%%%%%%%%%%%

%-----------------------------------------------------------------------------
% INTRODUCTION
%-----------------------------------------------------------------------------
\section{Introduction}
\label{sec:introduction}

This document is intended to be both an example of the Polimi \LaTeX{} template for the Executive Summary
of your thesis, as well as a short introduction to its use.

The Executive Summary is required only
if the thesis has been assigned to a reviewer (\textit{controrelatore})
for an independent evaluation of its quality, scientific/technical relevance and original contribution.

\section{Guidelines}
\label{sec:guidelines}

The Executive Summary is a critical overview of your thesis
with a focus on the main achievements that have emerged from your research.

The Executive Summary should be organized in sections/paragraphs
in order to better highlight the major points of your work.
The length should range from four to six pages depending on the length of the thesis manuscript.
Keep the Executive Summary concise enough to be effective but long enough to allow it to be complete.
It should be written after completing the thesis manuscript as a stand-alone independent document
of sufficient clarity and detail to ensure that the reader can figure out the overall objectives,
the methodology employed and the results/impact of your research.

In writing the Executive Summary, keep in mind that it is not an abstract, it is not a preface,
and it is not a random collection of highlights.
With a few exceptions, do not simply cut and paste whole sections or paragraphs of the thesis manuscript
into a disorganized and cluttered Executive Summary.
You should reorganize information to be informative as well as concise.

The Executive Summary could contain a few important equations related to your work.
It could also include the most relevant figures and tables taken or elaborated from the thesis manuscript.

You should also include in the Executive Summary the very essential bibliography of your study.
The number of selected references should range from three to five depending on the type of work.

The Executive Summary should contain a final section reporting the main conclusions drawn from your research.

\section{Sections and subsections}
\label{sec:sec_and_subsec}
It is convenient to organize the Executive Summary of your thesis into sections and subsections. 
If necessary, subsubsections, paragraphs and subparagraphs can be also used. 
A new section or subsection can be included  with the commands
\begin{verbatim}
\section{Title of the section}
\end{verbatim}
\begin{verbatim}
\subsection{Title of the subsection}
\end{verbatim}
It is recommended to give a label to each section by using the command
\begin{verbatim}
\label{sec:section_name}%
\end{verbatim}
where the argument is just a text string that you'll use to reference that part
as follows: \textit{Section~\ref{sec:sec_and_subsec} contains \sc{SECTIONS AND SUBSECTIONS}  \dots}.\\

%-----------------------------------------------------------------------------
% EQUATIONS AND FIGURES
%-----------------------------------------------------------------------------
\section{Equations, Figures, Tables and Algorithms}
\label{sec:equations_and_figures}
All Figures, Tables and Algorithms have to be properly referred in the text.
Equations have to be numbered only if they are referred in the text.
\subsection{Equations}
\label{sec_equations}
A few important equations related to your work might be reported in the Executive Summary. For example, the Maxwell's equations read:
\begin{subequations}
    \label{eq:maxwell}
    \begin{align}[left=\empheqlbrace]
    \nabla\cdot \bm{D} & = \rho, \label{eq:maxwell1} \\
    \nabla \times \bm{E} +  \frac{\partial \bm{B}}{\partial t} & = \bm{0}, \label{eq:maxwell2} \\
    \nabla\cdot \bm{B} & = 0, \label{eq:maxwell3} \\
    \nabla \times \bm{H} - \frac{\partial \bm{D}}{\partial t} &= \bm{J}. \label{eq:maxwell4}
    \end{align}
\end{subequations}

Equation~\eqref{eq:maxwell} is automatically labeled by \texttt{cleveref},
as well as Equation~\eqref{eq:maxwell1} and Equation~\eqref{eq:maxwell3}.
Thanks to the \verb|cleveref| package, there is no need to use \verb|\eqref|.

\subsection{Figures}
\label{sec:figures}
To include Figures in your text you can use \texttt{TikZ} for high-quality hand-made figures \cite{tikz},
or just include them with the command
\begin{verbatim}
\includegraphics[options]{filename.xxx}
\end{verbatim}
where xxx is the format (\verb|.png|, \verb|.jpg|, \verb|.eps|, \dots).
An example is shown in Figure~\ref{fig:quadtree}.
\begin{figure}[H]
    \centering
    \includegraphics[width=0.3\textwidth]{logo_polimi_scritta.eps}
    \caption{Caption of the Figure.}
    \label{fig:quadtree}
\end{figure}

\subsection{Tables}
\label{subsec:tables}

Within the environments \texttt{table} and  \texttt{tabular} you can create very fancy tables like the one shown in Table~\ref{table:example}.
\begin{table}[H]
    \caption*{\textbf{Example of Table}}
    \centering 
    \begin{tabular}{|p{3em} c c c |}
    \hline
    \rowcolor{bluePoli!40}
     & \textbf{column1} & \textbf{column2} & \textbf{column3} \T\B \\
    \hline \hline
    \textbf{row1} & 1 & 2 & 3 \T\B \\
    \textbf{row2} & $\alpha$ & $\beta$ & $\gamma$ \T\B\\
    \textbf{row3} & alpha & beta & gamma \B\\
    \hline
    \end{tabular}
    \\[10pt]
    \caption{Caption of the Table.}
    \label{table:example}
\end{table}

\subsection{Algorithms}
\label{subsec:algorithms}

Pseudo-algorithms can be written in \LaTeX{} with the \texttt{algorithm} and \texttt{algorithmic} packages.
One example follows.
\begin{algorithm}[H]
\label{alg:example}
\caption{Name of the Algorithm}
\label{alg:var}
\label{protocol1}
\begin{algorithmic}[1]
\STATE Initial instructions
\FOR{$for-condition$}
\STATE{Some instructions}
\IF{$if-condition$}
\STATE{Some other instructions}
\ENDIF
\ENDFOR
\WHILE{$while-condition$}
\STATE{Some further instructions}
\ENDWHILE
\STATE Final instructions
\end{algorithmic}
\end{algorithm} 

\section{Some further useful recommendations}

Theorems and Propositions have to be formatted as follows:
\begin{theorem}
\label{a_theorem}
Write here your theorem. 
\end{theorem}
\textit{Proof.} If useful you can report here the proof.
\vspace{0.3cm} % Insert vertical space

How to write propositions:
\begin{proposition}
Write here your proposition.
\end{proposition}
\vspace{0.3cm} % Insert vertical space

How to insert itemized lists:
\begin{itemize}
    \item first item;
    \item second item.
\end{itemize}
How to insert numbered lists:
\begin{enumerate}
    \item first item;
    \item second item.
\end{enumerate}

%-----------------------------------------------------------------------------
% HOW TO CITE BIBLIOGRAPHY
%-----------------------------------------------------------------------------
\section{Bibliography}
\label{sec:bibliography}
The Executive Summary should contain the very essential bibliography of your study.
It is suggested to use the BibTeX package \cite{bibtex} and save the bibliographic references
in the file  \verb|bibliography.bib|.

%-----------------------------------------------------------------------------
% CONCLUSION
%-----------------------------------------------------------------------------
\section{Conclusions}
A final section containing the main conclusions of your research/study have to be inserted here.

%---------------------------------------------------------------------------
%  ACKNOWLEDGEMENTS 
%---------------------------------------------------------------------------
\section{Acknowledgements}
Here you might want to acknowledge someone.

%---------------------------------------------------------------------------
%  BIBLIOGRAPHY
%---------------------------------------------------------------------------
% Remember to insert here only the essential bibliography of your work
\bibliography{bibliography.bib} % automatically inserted and ordered with this command 

\end{document}